%
% METU Institute of Natural and Applied Sciences Thesis example 
%
% Edited and Commented by Utku Erdoğdu 2013
% Modified for IAM - Institute of Applied Mathematics
% Graduate School of Applied Mathematics
% Edited and Commented by Omur Ugur 2017-2018
%
% Please read the explanations so that you can customize the document		
%
% Files needed by this document:
% metu.cls 
% metu11.def (if you will use 11pt fonts) 
% metu12.def (if you will use 12pt fonts)
% metu10.def (if you will use 10pt fonts)
%
% Possible Options Here:
%
% oneandhalf, double, single : Line spacing used in the thesis. 
% Default and institute preference is
% single.
%
% 10pt, 11pt, 12pt : Font size Default is 10pt, which is institue choice. 
% 
% pntr, pntc, pnbt : Page number position. 
% Options are top center, top right or bottom. 
% Default and institute preference is page numbers at bottom. 
% When page numbers are at the top bottom margins are skewed.
% 
% chaproman, chaparabic: Chapter numbering format. 
% Options are roman numbers and arabic numbers.
% Default is roman, institute prefers arabic
%
% oneside, twoside : Printing style. Default is twoside. 
% In this style chapters and (almost all) preliminary pages begin from 
% odd numbered pages.
%
% tr, eng : Document language. This is useful if you want to translate 
% your thesis into Turkish. 
% Then you give the option tr and use \ifturkish. . .\else. . .\fi  
% whenever you want 
% to do something only for Turkish or only for English. Default is eng. 
% IMPORTANT!! : For official institute documents you should not use this option. 
% The Turkish format is only supplied for custom translations.
%
% ceng,aee,arme.. : You can use the abbreviated form of your department here 
% and there is no further need to define the department name below. 
% If your department name is not among the below list of defined
% departments, you should use \department and \turkishdepartment macros 
% to define the name of your department.
%
% Defined Departments and Abbreviations:
% (may not work after modification for IAM)
% --------------------------------------
% Computer Engineering : ceng
% Aerospace Engineering : aee
% Archaeometry : arme
% Architecture : arch
% Biochemistry : bch
% Biology : biol
% Biomedical Engineering : bme
% Biotechnology : btec
% Building Science : bs
% Cement Engineering : ceme
% Chemical Engineering : che
% Chemistry : chem
% City and Regional Planning : crp
% City Planning : cp
% Civil Engineering : ce
% Computational Design and Fabrication Technologies in Architecture : arcd
% Computer Education and Instructional Technology : cte
% Design Research for Interaction : iddi
% Earthquake Studies : eqs
% Earth System Science : ess
% Electrical and Electronics Engineering : ee
% Engineering Management : em
% Engineering Sciences : es
% Environmental Engineering : enve
% Food Engineering : fde
% Geodetics - Geographical Information Technologies : ggit
% Geological Engineering : geoe
% Hydrosystems Engineering : he
% Industrial Design : id
% Industrial Engineering : ie
% Mathematics : math
% Mechanical Engineering : mech
% Metallurgical and Materials Engineering : mete
% Micro and Nanotechnology : mnt
% Mining Engineering : mine
% Operational Research : or
% Petroleum and Natural Gas Engineering : pete
% Physics : phys
% Polymer Science and Technology : pst
% Regional Planning : rp
% Restoration : rest
% Secondary Science and Mathematics Education : ssme
% Software Engineering : se
% Statistics : stat
% Structural Mechanics : st
% 
% phd, ms : Degree Received. Ph.D. or M.S. Default is M.S.
%
%
% Defined for use of IAM:
% acsc, cryp, fm, sc
%
% End of Options


% Use one of the following \documentclass with options of metu_iam.cls
\documentclass[chaparabic,sc,ms,12pt,oneandhalf]{metu_iam} 
%\documentclass[chaparabic,fm,ms,11pt,oneandhalf]{metu_iam} % preferred by IAM
%\documentclass[chaparabic,acsc,ms,10pt,single]{metu_iam} % preferred by the FBE
%\documentclass[chaparabic,cryp,phd,12pt,double]{metu_iam}

% You can delete next line If your thesis does not have an appendix
\usepackage{appendix}

%
% Personal Information 
% ----------------------------
%
% Please check this part and fill in information about your thesis
%
% Name and Surname
% be careful with the uppercase letters and Turkish characters
% following name should work for both UPPERCASE and LOWERCASE names
\author{UĞURCAN ÖZALP}
% Thesis Title English and Turkish
\title{Bipedal Robot Walking by Reinforcement Learning in Partially Observed Environment}
\turkishtitle{Pekiştirmeli Öğrenme Yöntemleriyle Kısmi Gözlenebilir Ortamda Çift Bacaklı Robotun Yürütülmesi}
% Department : English and Turkish
%
% Some of the departments are pre-defined, you need not redeclare them. 
% You can use them by just giving an option to \documentclass. 
% See documentation for options above. If you will define your 
% department here do not use ``Department'' or ``Bölümü'' words.
%\department{Computer Engineering}
%\turkishdepartment{Bilgisayar Mühendisliği}
%
% Date : You should indicate the month of your thesis defence in English.
% Default is this month
%
\date{August 2021} % do NOT use comma
%
% Approval Page Details
% --------------------------
% For each command you can give the title as optional parameter enclosed in [ ]
% This also handles the Turkish titles if you're planning 
% to produce Turkish version of the document. 
% If you'll hard code the title, you need 
% to use turkish version of each command after the command itself
% 
% prof : Prof. Dr.
% assocprof : Assoc. Prof. Dr.
% assistprof : Assist. Prof. Dr.
% dr : Dr.
%
% Director of Institute
\director[prof]{A. Sevtap Selçuk-Kestel}
% Head of Department
\headofdept[prof]{Hamdullah Yücel}
%
% Supervisor : English and Turkish
\supervisor[prof]{Ömür Uğur}
%\cosupervisor[assistprof]{Noname NoSurname} % suppress/comment if you don't have cosupervisor
% \turkishsupervisor{  } %if you will hard-code the academic title
%
% Affiliation of Supervisor in English and possibly in Turkish
\departmentofsupervisor{Scientific Computing, METU}
% \turkishcosupervisor{Prof. Dr. Reda Alhajj} %if you will hard-code the academic title
% Affiliation of Co-Supervisor
% You can just delete/comment the next line if you don't have a co-supervisor
%\departmentofcosupervisor{Computer Engineering Dept., Bilkent Uni.}
%
% Committee Members
% In general members are sorted according to their academic titles; however
% new modifications indicate
%
% The Chair (1)
% Supervisor (2)
% Co-supervisor, if any (3)
% Academic Titles (4,5) or (3-5)
% 
% IMPORTANT:  All affiliatons should fit in a single line
% If affiliation line is broken into two lines you should shorten the affiliation by using 
% abbrevations or any other means
%
\committeememberi[assocprof]{Ümit Aksoy}
\affiliationi{Mathematics, Atılım University}
% Second committee member is always your supervisor
\committeememberii[prof]{Ömür Uğur}
\affiliationii{Scientific Computing, METU}
% If you are an M.Sc. student and your Co-Supervisor is in your 
% examination committee, then third committee member is always your co-supervisor
%
% IMPORTANT: If you are Ph.D. student your co-supervisor cannot be in your 
% examination committee.
\committeememberiii[assistprof]{Önder Türk}
\affiliationiii{Scientific Computing, METU}

% Fourth committee member
% If you have only three (3) Committee Members;
% this can ONLY be if this is a M.Sc thesis
% then uncomment \committeeivfalse so that
% 4th and 5th members do NOT count at all:
\committeeivfalse
% Otherwise, you need to fill in the 4th and the 5th members
\committeememberiv[assocprof]{Ceylan Yozgatlıgil}
\affiliationiv{Statistics, METU}
% Fifth committee member is a MUST if Fourth committee member is
\committeememberv[prof]{Ayhan Aydın}
\affiliationv{Mathematics, Atılım University}
%
% Keywords : English & Turkish, Comma seperated
\keywords{deep reinforcement learning, partial observability, robot control, actor-critic methods, long short term memory, transformer}
\anahtarklm{pekiştirmeli derin öğrenme, kısmi gözlemlenebilirlik, robot kontrolü, aktör-eleştirmen metodları, uzun kısa süreli bellek, transformatör}

%
% Abstract in English
%
\abstract{
% either write abstract here or simply input
Deep Reinforcement Learning methods on mechanical control are successful on many environments and used instead of traditional optimal and adaptive control methods on some complex cases. However, Deep Reinforcement Learning algorithms do still have challenges. One is control on partially observable environments. When an agent is not informed well about the environment, it must recover information from past observations. In this thesis, walking of Bipedal Walker (OpenAI GYM) environment is studied by continious actor-critic reinforcement learning algorithm Twin Delayed Deep Determinstic Policy Gradient. Environment is partially observable because walker is not able to see behind. Several neural architectures are implemented. First one is Residual Feed Forward Neural Network under the observable environment assumption, while second and third ones are Long Short Term Memory and Transformer using observation history as input to recover hidden state since environment is assumed to be partially observable.



 % filename: abstract.tex
}
%
% Turkish Abstract
%
\oz{
% either write öz here or simply input
google translate şimdilik -> Mekanik kontrol üzerine Derin Takviyeli Öğrenme (DRL) yöntemleri birçok ortamda başarılıdır ve bazı karmaşık durumlarda geleneksel optimal ve uyarlanabilir kontrol yöntemleri yerine kullanılır. Ancak, DRL algoritmalarının hala zorlukları vardır. Birincisi, kısmen gözlemlenebilir ortamlarda kontroldür. Bir temsilci çevre hakkında yeterince bilgilendirilmediğinde, geçmiş gözlemlerden bilgileri kurtarmalıdır. Bu tezde, Bipedal Walker (OpenAI GYM) ortamının DRL kontrolü, sürekli aktör-eleştirmen algoritması Twin Delayed Deep Determinstic Policy Gradient (TD3) ile DRL tarafından incelenmiştir. Çevre kısmen gözlemlenebilir olduğundan, birkaç sinir mimarisi uygulanmaktadır Birincisi, gözlemlenebilir ortam varsayımı altında ileri beslemeli sinir ağı iken, ikincisi ve üçüncüsü, kurtarmak için girdi olarak son 16 zaman adımı gözlemini kullanan Transformer ve Uzun Kısa Süreli Bellek (LSTM) gizli durum, çünkü çevrenin kısmen gözlemlenebilir olduğu varsayılır. % filename: oz.tex
} 
%
% Dedication 
\dedication{\textit{ For anyone who is curious to read. % better to write here as it is short enough
}}
%
%
% Acknowledgements   
\acknowledgments{
% either write acknowledgments here or simply input
I would like to thank my thesis supervisor Prof. Dr. Ömur U\u{g}ur whose insightful comments and suggestions were of inestimable value for my study. His willingness to give his time and share his expertise has paved the way for me. \\

Special thanks also go to my friend Mehmet Gökçay Kabataş whose opinions and information have helped me very much throughout the production of this study. \\

I would also like to express my gratitude to my family for their moral support and warm encouragements. Especially, I would like to show my greatest appreciation to Serpil Sökmen, who provides me to come these days. \\

Lastly, I would like to thank my partner Dilara Bayram for supporting me in long study days. \\
 % filename: acknowledgments.tex
}
%
% End of Personal and Introductory Information
%%%%%%%%%%%%%%%%%%%%%%%%%%%%%%%%%5

%%% !!! These are most probably you need for effective use of metu_iam.cls
\usepackage{epigraph} % Added by Ugurcan Ozalp
\usepackage{graphicx}
%\graphicspath{ {./figures/} } % if it helps uncomment
\usepackage{ifpdf} % if you use pdflatex (generally I do not suggest,
                   % but it depends on your graphic files)
\ifpdf
\usepackage{epstopdf}
\usepackage[pdftex,bookmarks=true,bookmarksnumbered=false,breaklinks=true]{hyperref}
\pdfadjustspacing=1
\usepackage[pdftex]{thumbpdf}
%\usepackage{pdfpages} % needed if you wish to include external pdf page (NOT suggested)
\DeclareGraphicsExtensions{.pdf,.png,.jpg}
%\usepackage[pdftex,linktocpage=true,breaklinks=true]{hyperref}
\else
% you may choose either
%\usepackage[ps2pdf,bookmarks=true,bookmarksnumbered=true,breaklinks=true]{hyperref} % or
\usepackage[ps2pdf,linktocpage=true,bookmarks=true,bookmarksnumbered=false,breaklinks=true]{hyperref}
\usepackage[ps2pdf]{thumbpdf}
\DeclareGraphicsExtensions{.eps}
%\usepackage[ps2pdf,linktocpage=false,breaklinks=true]{hyperref}
\fi
\usepackage[all]{hypcap} 

%
%%% Necessary packages to compile this Template for IAM
\usepackage{xcolor}
\usepackage{fancyvrb}
\usepackage{xy} 
\usepackage{listings}
\usepackage{newfloat}
% Most likely, you will need these package of AMS
% These are strongly remommended
\usepackage{amsmath,amsfonts}
\usepackage{amsthm,amssymb}
% unfortunately for this template we need
\usepackage{caption} % although gives warning of unsopported package.
\usepackage{subcaption} % although gives warning of unsopported package.
\usepackage[ruled,vlined]{algorithm2e} % package for algorithms.
\SetAlFnt{\small}
% Extra
\usepackage{slashbox}
%%% End of Necessary Packages
%

%%% Extra packages if you need
%\usepackage{rotating}
%\usepackage{booktabs}
%\usepackage{pifont}

%%%
%%% Verbatim/Listings Environments
%%%
% In most cases, we need to present coding/listing of Matlab or Python
% or other programming languages.
% Unfortunately, lstlistings does not work metu_iam.cls !!!
% It is better to use a workaround instead:
%
%%% Suggestion: minted.sty (however not default here, due to the following.)
%%% HERE IS WHAT YOU HAVE TO DO FIRST: 
%%% You must invoke LaTeX with the -shell-escape flag;
%%% and you need to have python as well as pigments installed.
%uncomment%   \usepackage[newfloat]{minted} % This package supports many languages (and their color schema). 
% if you are using linux, no problem (in most cases)
% if you are using Windows, then
% 1. download and install Python for Windows on https://www.python.org/downloads/windows/
% 2. open a command prompt in Windows and run on the prompt (C:\Users\yourname>)
%   (a) python -m pip install -U pip setuptools
%   (b) easy_install pygments
%    make sure that the executables are reachable within the PATH environment variable
% 3. restart Windows
% Now minted package can be used. 
% Beware: -shell-escape flag must be used when latex or pdflatex compiler is invoked.
% Also, if you decide to use minted comment the definition of the listing environment below
%
%%% Alternative Approach (however without colours) is
%%% to use Verbatim Environment
% for this Template for instance we define,
% similar to the 'listing' environment in minted.sty :)
% if you don't want to use this new environment, 
% or, if you are using minted.sty, you should/must comment the lines:
%comment%   
\DeclareFloatingEnvironment[fileext=lol,placement=tbp]{listing}
%comment%   
\newcommand{\listingscaption}{Listing}
 %\floatname{listing}{\listingscaption} % if float, instead of newfloat, is loaded
%\newcommand{\listoflistingscaption}{List of Listings} % will never be used
%\providecommand{\listoflistings}{\listof{listing}{\listoflistingscaption}}  % will never be used
%
%%% End of Verbatim/Listing Environments

% include any other definitions, abbreviations for LaTeX commands
% you might need. Use it after carefully defining 
% your own commands and definitions (even packages)
% some handy shortcut commands

\newcommand{\sA}{{\mathcal A}}
\newcommand{\sB}{{\mathcal B}}
\newcommand{\sC}{{\mathcal C}}
\newcommand{\sD}{{\mathcal D}}
\newcommand{\sE}{{\mathcal E}}
\newcommand{\sF}{{\mathcal F}}
\newcommand{\sG}{{\mathcal G}}
\newcommand{\sH}{{\mathcal H}}
\newcommand{\sI}{{\mathcal I}}
\newcommand{\sJ}{{\mathcal J}}
\newcommand{\sK}{{\mathcal K}}
\newcommand{\sL}{{\mathcal L}}
\newcommand{\sM}{{\mathcal M}}
\newcommand{\sN}{{\mathcal N}}
\newcommand{\sO}{{\mathcal O}}
\newcommand{\sP}{{\mathcal P}}
\newcommand{\sQ}{{\mathcal Q}}
\newcommand{\sR}{{\mathcal R}}
\newcommand{\sS}{{\mathcal S}}
\newcommand{\sT}{{\mathcal T}}
\newcommand{\sU}{{\mathcal U}}
\newcommand{\sV}{{\mathcal V}}
\newcommand{\sW}{{\mathcal W}}
\newcommand{\sX}{{\mathcal X}}
\newcommand{\sY}{{\mathcal Y}}
\newcommand{\sZ}{{\mathcal Z}}

%%%%%%%%%%%%%%%%%%%%%%%%%%%%%%%%%%%%%%%%%%%%%%

\newcommand{\A}{{\mathbb A}}
\newcommand{\B}{{\mathbb B}}
\newcommand{\C}{{\mathbb C}}
\newcommand{\D}{{\mathbb D}}
\newcommand{\E}{{\mathbb E}}
\newcommand{\F}{{\mathbb{F}}}
\newcommand{\G}{{\mathbb G}}
\newcommand{\HH}{{\mathbb H}}
\newcommand{\I}{{\mathbb I}}
\newcommand{\J}{{\mathbb J}}
\newcommand{\M}{{\mathbb M}}
\newcommand{\N}{{\mathbb N}}
\renewcommand{\P}{{\mathbb P}}
\newcommand{\Q}{{\mathbb Q}}
\newcommand{\R}{{\mathbb R}}
\newcommand{\T}{{\mathbb T}}
\newcommand{\U}{{\mathbb U}}
\newcommand{\V}{{\mathbb V}}
\newcommand{\W}{{\mathbb W}}
\newcommand{\X}{{\mathbb X}}
\newcommand{\Y}{{\mathbb Y}}
\newcommand{\Z}{{\mathbb Z}}

%%%%%%%%%%%%%%%%%%%%%%%%%%%%%%%%%%%%%%%%%%%%%%

\newcommand{\dsp}{\displaystyle}
% mathematical operators, such as \min, \max, \lim, \log, etc...
\DeclareMathOperator{\argmax}{\textup{argmax}}
% or similar
\newcommand{\diag}{\mathop{\mbox{\textup{diag}}}}

% to be consistent within the text define (your own commands):
\newcommand{\norm}[1]{\left\Vert {#1} \right\Vert}
\newcommand{\abs}[1]{\left\vert {#1} \right\vert}
\renewcommand{\Re}[1]{\mathcal{R}e\left( {#1} \right)}
\renewcommand{\Im}[1]{\mathcal{I}m\left( {#1} \right)}
\newcommand{\poincare}{Poincar\'e{}}
\newcommand{\re}[1]{\ensuremath{\mathcal{R}e\left(#1\right)}}

% to be consistent in referring items, chapters, sections, lemmas, theorems, etc...
\newcommand{\thmref}[1]{Theorem~\ref{#1}}
\newcommand{\lemref}[1]{Lemma~\ref{#1}}
\newcommand{\chapref}[1]{Chapter~\ref{#1}}
\newcommand{\secref}[1]{Section~\ref{#1}}
\newcommand{\tabref}[1]{Table~\ref{#1}}
\newcommand{\figref}[1]{Figure~\ref{#1}}
% \eqref is most probably defined, in case you need uncomment
%\renewcommand{\eqref}[1]{Eq.~\ref{#1}}

%%% Verbatim package's options to be consistent within the document
% although this Template includes many inconsistent ways
% for illustration purposes.
% depending on your needs, spacing, etc. modify below or comment 
\fvset{baselinestretch=1.2, frame=lines, framerule=1pt, tabsize=2, 
	numbers=left, fontsize=\footnotesize, xleftmargin=25pt, xrightmargin=10pt} 
	 % filename: myDefinitions.tex

\theoremstyle{definition}
\newtheorem{definition}{Definition}[section]

\begin{document}
% Preliminaries
\begin{preliminaries}
% If you are willing to use any custom stuff before Chapters, put it here
% Such as List of Abbreviations
% Check the abbreviations.tex for a template list of abbreviations
\begin{theglossary}{LONGESTABBRV}
\item[AI] Artificial Intelligence
\item[ML] Machine Learning
\item[DL] Deep Learning
\item[RL] Reinforcement Learnin
\item[DRL] Deep Reinforcement Learning
\item[FFNN] Feed Forward Neural Network
\item[RFFNN] Residual Feed Forward Neural Network
\item[RNN] Recurrent Neural Network
\item[LSTM] Long Short Term Memory

\end{theglossary}
 % filename: abbreviations.tex
% End of Preliminaries
\end{preliminaries}
%   
%%% Latex Content Goes Here 
% 
%%% Suggestion is that input your LaTeX files 
% including chapters and sections

\chapter{CONCLUSION AND FUTURE WORK}
\label{chap:conclusion}

\section{Conclusion}

In this thesis, bipedal robot walking is investigated by deep  reinforcement learning methods. 

Twin Delayed Deep Deterministic Policy Gradient (TD3) 

As stated in previous chapters, most of the real world environments are partially observable. In Bipedal-Walker-Hardcore, the environment is also partially observable since agent cannot observe behind of it lacks of acceleration sensors which is better to have for controlling mechanical systems. Therefore, we used Long Short Term Memory and Transformer Neural Networks to capture more information from past observations compared to single instant observation. Along with them, we also implemented Residually connected Feed Forward Neural Network using single instant observation. \\

We were expecting to have better results with xxyy.

\section{Future Work}


\section{Problem Statement: Bipedal Walker Robot Control}
\label{sec:problem_statement}


\subsection{OpenAI Gym and Bipedal Walker Environment}
\label{ssec:gym_bipedal}

OpenAI Gym \cite{brockman_openai_2016} is open source framework, 
containing many environments to service development of 
reinforcement learning algorithms. 

BipedalWalker environments~\cite{noauthor_bipedalwalker-v2_2021, noauthor_bipedalwalkerhardcore-v2_2021} are part of Gym environment library. 
One of them is classical version where the terrain is relatively smooth, while other one is hardcore version which contains ladders, stumps and pitfalls in terrain. 
Those environments have continious action and observation space. 
For both settings, the task is to move forward the robot as much as possible. 
Snapshots for both environments are in \figref{fig:bipedal_walkers}.
\begin{figure}
	\begin{subfigure}{.5\textwidth}
		\centering
		\includegraphics[width=0.9\linewidth]{figures/bipedal/classic.png}
		\caption{BipedalWalker-v3 Snapshot}
		\label{fig:bipedal_walker_classic}
	\end{subfigure}
	\begin{subfigure}{.5\textwidth}
		\centering
		\includegraphics[width=0.9\linewidth]{figures/bipedal/hardcore.png}
		\caption{BipedalWalkerHardcore-v3 Snapshot}
		\label{fig:bipedal_walker_hardcore}
	\end{subfigure}
	\caption{Bipedal Walkers Snapshots}
	\label{fig:bipedal_walkers}
\end{figure}

Locomotion of the Bipedal Walker is difficult control problem due to following reasons. 
\begin{itemize}
	\item \textbf{Nonlinearity}: The dynamics is nonlinear, unstable and multimodal. 
	Dynamical behavior of robot changes for different situations 
	like ground contact, single leg contact and double leg concact. 
	\item \textbf{Uncertainity}: The terrain where robot walks may vary. 
	Designing a controller for all types of terrain is difficult.
	\item \textbf{Partially Observability}: The robot observes 
	ahead of it with lidar measurements and cannot observe behind. 
	In addition, it lacks of acceleration sensors.
\end{itemize}
These difficulties make hard to implement analytical methods for control task. 
RL approach is better to tackle first 2 one. 
For partial observability problem, more elegant solution is required. 
This is done by creating a belief state from past observations to inform agent. 
Agent uses this belief state to choose how to act. 
If belief state is evaluated sufficiently, 
this increases performance of control better.
However, relying on observations is also possible for control, 
and this may be enough sometimes if advanced control is not required. 


\subsection{Deep Learning Library: PyTorch}
\label{dl_pytorch}
PyTorch is an open source library developed by Facebook's AI Research lab (FAIR)~\cite{paszke_pytorch_2019}. 
It is based on Torch library~\cite{collobert_torch7_2011} and has Python and C++ interface. 
It is an automatic differentation library with accelerated math operations backed by graphical processing units (GPUs). 
This is what a deep learning library requires. 
And the clean pythonic syntax made it most famous deep learning tool among researches. 

\section{Proposed Methods and Contribution}
\label{sec:proposed_methods}

Partially observable environments are always a hard work for reinforcement learning algorithms. 
In this work, the walker environment is assumed to be fully observable environment at first. 
A Residual Feed-Forward Neural Network architecture is proposed to control 
the robot under fully observability assumption due to the fact that no memory is used during decision making. 
Then, the environment is assumed to be partially observable. 
In order to recover belief states, Long Short Term Memory (LSTM) and 
Transformer neural networks are proposed using fixed number of 
past observations (6 and 12 in our case) during decision making. 

LSTM is used in many deep learning applications including sequential data. 
It is a variant of Recurrent Neural Networks (RNN) and a good candidate for RL algorithms to be applied in partially observable environments. 

Transformer is developed to handle sequential data as RNN models do. 
However, it processes the whole sequence at the same time, while RNN processes the sequence in order. 
Transformers are commonly used in Natural Language Processing (NLP) thanks to major performance improvements over RNN variants, but this is not the case for Reinforcement Learning, yet.

In order to handle the reward sparsity problem, reward function is redesigned in this thesis. Also, an exploration strategy is formed so that the agent both explores and learns sufficiently well. 

In this thesis, Twin Delayed Deep Deterministic Policy Gradient (TD3) and Soft Actor Critic (SAC) are used to solve our environment as RL algorithm. 
TD3 is a deterministic RL method with additive exploration noise, and it is improved version of Deep Deterministic Policy Gradient (DDPG).
SAC is a stochastic type of RL method with adaptive exploration. 
It adjusts how much to explore depending on observations and rewards.

\section{Related Work}
\label{sec:related_work}

Reinforcement Learning methods are used in many mechanical control tasks 
such as autonomus driving \cite{pan_virtual_2017, shalev-shwartz_safe_2016, sallab_deep_2017, wang_deep_2019} 
and autonomus flight \cite{kopsa_reinforcement_2018, abbeel_application_2006, santos_experimental_2012}.

Rastogi \cite{rastogi_deep_2017} used Deep Deterministic Policy Gradient (DDPG) algorithm to walk 
their physical bipedal walker robot along with simulation environment. 
They concluded that DDPG is infeasible to control walker robot 
since it requires long time for convergence. 
Kumar et al. \cite{kumar_bipedal_2018} also used DDPG to perform 
robot walking in 2D simulation environment. 
Their agent converged in approximately 25,000 episodes. 
Song et al. \cite{song_recurrent_2018} pointed out the partial observability problem of bipedal walker, 
using Recurrent Deep Deterministic Policy Gradient (RDDPG)~\cite{heess_memory-based_2015} algorithm 
and acquired better results than original Deep Deterministic Policy Gradient (DDPG) algorithm. 

Fris \cite{fris_landing_2020} used Twin Delayed Deep Deterministic Policy Gradient (TD3) 
using LSTM for their quadrocopter landing task. 
Fu et al. \cite{fu_when_2020} used vanilla RNN with attention mechanism 
using TD3 for car driving task, but not explicit Transformer. 
They reported that their method outperformed 7 baselines. 
Upadhyay et al. \cite{upadhyay_transformer_2019} used all of feed forward network, 
LSTM an original Transformer architectures for balancing pole 
on a cart from Cartpole environment of Gym, and Transformer yield worst results among three architectures.

%%% SAC?
\section{Outline of the Thesis}
\label{sec:outline}
This thesis consists of five chapters. 
In Chapter 2, we discuss the theory of Reinforcement Learning and introduce the methods used in this thesis.
In Chapter 3, we explain the theory of Neural Networks and Deep Learning along with architectures which are designed to process sequential data. 
In Chapter 4, Bipedal Walker environments are presented, neural networks and RL algoritmhs are proposed, results are summarized and discussed.
In the last chapter, thesis is concluded by discussing obtained results and possible future work is outlined along with the future of RL.

\chapter{CONCLUSION AND FUTURE WORK}
\label{chap:conclusion}

\section{Conclusion}

In this thesis, bipedal robot walking is investigated by deep  reinforcement learning methods. 

Twin Delayed Deep Deterministic Policy Gradient (TD3) 

As stated in previous chapters, most of the real world environments are partially observable. In Bipedal-Walker-Hardcore, the environment is also partially observable since agent cannot observe behind of it lacks of acceleration sensors which is better to have for controlling mechanical systems. Therefore, we used Long Short Term Memory and Transformer Neural Networks to capture more information from past observations compared to single instant observation. Along with them, we also implemented Residually connected Feed Forward Neural Network using single instant observation. \\

We were expecting to have better results with xxyy.

\section{Future Work}


\section{Reinforcement Learning and Optimal Control}
\label{sec:rl_and_control}

Optimal control is a field of mathematical optimization, 
concerned by  finding control policy of a dynamical system (environment) for given objective. 
For example, objective might be total revenue for a company as system, 
minimal fuel burn for a car as system, or total production for a factory. 

RL is kind of naive subfield of Optimal Control. 
However, RL algorithms find policy (controller) by error minimization of objective from experience, 
while Optimal Control methods are concerned exact analytical optimal solutions based on dynamic model of environment and agent. 

Optimal Control methods are efficient and robust when mathematical model of environment is available, 
accurate enough and solvable for optimal controller. 
However, some real world problems usually do not exhibit all of these conditions. 
In such cases, Reinforcement Learning is an easier way to derive a control policy.

\section{Challenges}
\label{sec:chal}

The Reinforcement Learning Environment poses a variety of obstacles 
that we need to address and potentially make trade-offs among them~\cite{dulac-arnold_challenges_2019, sutton_reinforcement_1998}.

\subsection{Exploration Explotation Dilemma}

A RL agent is supposed to maximize rewards (explotation of knowledge) by observing the environment (exploration of environment). 
This gives rise to the exploration-exploitation dilemma that is the inevitable trade-off between them. 
Exploration is taking a range of acts to benefit about the consequences. 
Typically results in low immediate rewards and high rewards for the future. 
Explotation is taking action that has been learned. Typically results in high immediate rewards and low rewards in the future. 

\subsection{Generalization and Curse of Dimensionality}

A RL agent should also be able to generalize experiences to act on unseen situation before. 
This issue arises when state space and action space is high dimensional since experiencing all possibilities is impractical. 
This is solved by introducing function approximators. Deep Reinforcement Learning uses neural network as function approximator. 

\subsection{Delayed Consequences}

A RL agent should be aware reason of reward or punishment. 
Once it gets reward or punishment, it should be able to discriminate whether reward is caused by instant actions or past actions. 

\subsection{Partial Observability}

Partial observability is absence of all required observation to infer instant state. 
For instance, a driver does not know engine temperature or rotational speed of gears. 
Although driver is able to drive in that case, s/he would not be able to drive well on traffic in absence of rear view mirror or side mirror. 
In real world, most of systems are partially observable. 
This problem is usually tackled by incorprating observation history from agents memory in acting. 

\subsection{Safety of Agent}

Mechanical agents can kill or degrade themselves and their surroundings. 
This safety problem is important on both exploration stage and full operation. 
Simulation of environment is a good way to train agent with safety but causes incomplete learning 
due to inaccuracy compared to real environment. 

\section{Sequential Decision Making}
\label{sec:decision_making}

RL may also be considered as a stochastic control process in discrete time setting~\cite{sutton_reinforcement_1998}. 
At time $t$, the agent starts with state $s_t$ and observes $o_t$, 
then it takes an action $a_t$ according to its policy $\pi$ and obtains a reward $r_t$ at time $t$. 
Hence, a state transition to $s_{t+1}$ occurs as a consequence of the action and the agent gets the next observation $o_{t+1}$. 
History is, therefore, the ordered set of past actions, observations and rewards: $h_t=\{ a_0, o_0, r_0, ... a_t, o_t, r_t\}$. 
The state $s_t$ is a function of the history, i.e., $s_t=f(h_t)$, 
which represents the characteristics of environment at time $t$ as much as possible. 
The RL diagram is visualized in \figref{fig:rl_diagram}. 
\begin{figure}
	\centering
	\includegraphics[width=0.7\textwidth]{figures/ml_theory/RL_diagram.png}
	\caption{Reinforcement Learning Diagram}
	\label{fig:rl_diagram}
\end{figure}
\section{Markov Decision Process}
\label{sec:mdp}

Markov Decision Process (MDP) is a sequential decision making process with Markov property. 
It is represented as a tuple $(\mathcal{S},\mathcal{A},T,R,\gamma)$. 
Markov property means that the conditional probability distribution of the future state depends only on the instant state and action instead of the entire state/action history, so it is regarded as memoryless. 
In MDP setting, the system is fully observable which means that the states can be derived from instant observations; i.e., $s_t=f(o_t)$. 
Therefore, agent can decide an action based on only instant observation $o_t$ instead of what happened at previous times \cite{francois-lavet_introduction_2018}. MDP consists of the following:

\begin{description}
	\item[State Space $\mathcal{S}$] A set of all possible configurations of the system. 
	
	\item[Action Space $\mathcal{A}$]  A set of all possible actions of the agent. 
	
	\item[Model $T \colon \mathcal{S} \times \mathcal{S} \times \mathcal{A} \rightarrow \lbrack 0,1 \rbrack$] A function of how environment evolves through time, representing transition probabilities as $T(s'|s,a) = p(s'|s,a)$ 
	where $s' \in \mathcal{S}$ is the next state, $s \in \mathcal{S}$ is the instant state and $a \in \mathcal{A}$ is the action taken.
	
	\item[Reward Function $R \colon \mathcal{S} \times \mathcal{A} \rightarrow \mathbb{R}$] A function of rewards obtained from the environment. 
	At each state transition $s_t \rightarrow s_{t+1}$, a reward $r_t$ is given to the agent. 
	Rewards may be either deterministic or stochastic. 
	Reward function is the expected value of reward given the state $s$ and the action taken $a$, defined by:
	\begin{equation}
	R(s,a) = \mathbb{E}[r_t|s_t=s, a_t=a]. %\: \forall t = 0,1, ...
	\end{equation}
	
	\item[Discount Factor $\gamma \in \lbrack 0,1 \rbrack$] A measure of the importance of rewards in the future for the value function.
\end{description}

\section{Partially Observed Markov Decision Process}
\label{sec:pomdp}

In MDP, agent can recover full state from observations, i.e., $s_t=f(o_t)$. 
However, observation space is not enough to represent all information (states) about the environment sometimes. 
That means one needs more observations from the history, i.e., $s_t=f(o_t, o_{t-1}, o_{t-2}, ...)$.
In such cases, past and instant observations are used to filter out a belief state. 
It is represented as a tuple $(\mathcal{S},\mathcal{A},T,R,\mathcal{O},O,\gamma)$. 
In addition to MDP, it introduces observation space $\mathcal{O}$ and observation model $O$ \cite{francois-lavet_introduction_2018}: 

\begin{description}
	\item[Observation Space $\mathcal{O}$] A set of all possible observations of the agent.
	\item[Observation Model $O \colon \mathcal{O} \times \mathcal{S} \rightarrow \lbrack 0,1 \rbrack$] A function of how observations are related to the states, 
	representing observation probabilities as $O(o|s) = p(o|s)$ 
	where $s \in \mathcal{S}$ is the instant state and $o \in \mathcal{O}$ is the observation.
\end{description}

Since states are not observed directly, the agent needs to use the observations while deriving a control policy. 
\section{Policy and Control}
\label{sec:policy_control}

\subsection{Policy}

A policy defines how the agent acts according to the state of the environment. 
It may be either deterministic or stochastic: 

\begin{description}
	\item[Deterministic Policy $\mu \colon \mathcal{S} \rightarrow \mathcal{A}$]. 
	A mapping from states to actions.
	\item[Stochastic Policy $\pi \colon \mathcal{S} \times \mathcal{A} \rightarrow \lbrack 0,1 \rbrack$]. 
	A mapping from state-action pair to a probability value.
\end{description}

\subsection{Return}

At time $t$, return $G_t$ is a cumulative sum of the future rewards scaled by the discount factor $\gamma$: 
\begin{equation}
\label{eqn:return_dfn}
G_t = \sum_{i=t}^{\infty} \gamma^{i-t} r_i = r_t + \gamma G_{t+1}.
\end{equation}
Since the return depends on future rewards, it also depends on the  policy of the agent as it affects future rewards.

\subsection{State Value Function}

State Value Function $V^{\pi}$ is the expected return when policy $\pi$ is followed in future and it is defined by
\begin{equation}
V^{\pi}(s) = \mathbb{E}[G_t|s_t=s, \pi]. % \quad \forall t = 0,1, ...
\end{equation}
Optimal value function should return maximum expected return. 
The behavior is controlled by the policy. In other words, 
\begin{equation}
V^{*}(s) = \max_{\pi} V^{\pi}(s).
\end{equation}

\subsection{State-Action Value Function}

State-Action Value Function $Q^{\pi}$ is again the expected return when policy $\pi$ is followed in future, 
however, any action taken at the instant step:
\begin{equation}
Q^{\pi}(s,a) = \mathbb{E}[G_t|s_t=s, a_t=a, \pi]. %\quad \forall t = 0,1, ...
\end{equation}
Optimal state-action value function should yield maximum expected return for each state-action pair. Hence,
\begin{equation}
Q^{*}(s,a) = \max_{\pi} Q^{\pi}(s,a).
\end{equation}
Similarly, the optimal policy $\pi^*$ can be obtained by $Q^{*}(s,a)$. For stochastic policy, it is defined as follows: 
\begin{equation}
\label{eqn:policy_stochastic_q}
\pi^{*}(a|s) = 
\begin{cases}
1,   & \text{if  } a = \arg\max_{a} Q^{*}(s,a) \\
0,   & \text{otherwise  }
\end{cases} 
\end{equation}
For deterministic policy, 
\begin{equation}
\label{eqn:policy_deterministic_q}
\mu^{*}(s) = \arg\max_{a} Q^{*}(s,a)
\end{equation}

\subsection{Bellman Equation}

Bellman proved that optimal value function should satisfy following conditions~\cite{bellman_dynamic_2003}. 
\begin{equation}
\label{eqn:bellman_v}
V^{*}(s) = \max_{a} \Big\{ R(s,a) + \gamma \sum_{s'} T(s'|s,a) V^{*}(s') \Big\}
\end{equation}
\begin{equation}
\label{eqn:bellman_q}
Q^{*}(s,a) = R(s,a) + \gamma \max_{a'} \Big\{ \sum_{s'} T(s'|s,a) Q^{*}(s',a') \Big\}
\end{equation}

\section{Model Free Reinforcement Learning}
\label{sec:mf_rl}

Model based methods are based on solving bellman equation \ref{eqn:bellman_v}\ref{eqn:bellman_q} with given model $T$. On the other hand, Model Free Reinforcement Learning is suitable if environment model is not available but agent can experience environment by consequences of its actions. There are 3 main types.

\textbf{Value Based Learning}: Value functions are learned, then policy arises naturally from value function as shown in \ref{eqn:policy_stochastic_q} \ref{eqn:policy_deterministic_q}. Since argmax operation is used, this type of learning is suitable for problems where action space is discrete.

\textbf{Policy Based Learning}: Policy is learned directly, return values are used instead of learning a value function. Unlike value based methods, it is suitable for continious action spaces.

\textbf{Actor Critic Learning}: Both policy (actor) and value (critic) functions are learned simulatenously. It is also suitable for continious action spaces. 

\subsection{Q Learning}
Q Learning is a value based type of learning. It is based on optimizing $Q$ function using bellman equation \ref{eqn:bellman_q} \cite{watkins_technical_1992}. 

Assume that $Q$ function is parametrized by $\theta$. Target $Q$ value is estimated by bootstrapping estimate itself, as shown in  \eqref{eqn:q_target}. 
%
\begin{equation}
\label{eqn:q_target}
Y_t^Q = r_t + \gamma \max_{a'} Q(s_{t+1},a';\theta)
\end{equation}

At time $t$,  with state, action, reward, next state tuples ($s_t,a_t,r_t,s_{t+1}$), $Q$ values are updated by minimizing difference between target value and estimated value with respect to $\theta$ using numerical optimization methods.

\begin{equation}
\label{eqn:q_loss}
\mathcal{L}_t(\theta) = \big( Y_t - Q(s_t,a_t;\theta) \big) ^ 2
\end{equation}

\subsubsection{Deep Q Learning}

Deep Q Learning overcomes instability of Q Learning. When a nonlinear approximator is used, learning is unstable. Deep Q Learning introduces Target Network and Experience Replay \cite{mnih_human-level_2015, mnih_playing_2013}. 

\textbf{Target Network}: Target network is parametrized $\theta^-$. It is used to evaluate target value and not updated by loss minimization. It is updated at each fixed number of update step $C$ by Q network parameter $\theta$. Target value is obtained by using $\theta^-$.

\begin{equation}
\label{eqn:dqn_ntarget}
Y_t^{DQN} = r_t + \gamma \max_{a'} Q(s_{t+1},a';\theta^-)
\end{equation}

\textbf{Experience Replay}: Experience tuples are stored in dataset $\mathcal{D}$ as queue with fixed buffer size $N_{replay}$. At each iteration $i$, $\theta$ is updated by experiences uniformly subsampled from experience replay. It allows agent to learn from experiences multiple times. More importantly, sampled experiences are close to be independent and identically distributed if buffer size is high enough. This makes learning process more stable.  

\begin{equation}
\label{eqn:dqn_loss}
\mathcal{L}_i(\theta_i) = \mathbb{E}_{s,a,r,s'\sim U(\mathcal{D})}\Big[\big( Y^{DQN} - Q(s,a;\theta_i) \big) ^ 2 \Big]
\end{equation}

\textbf{Epsilon Greedy Exploration}: As stated in \chapref{sec:chal}, exploration is an important step for reinforcement learning algorithms. In discrete action space (finite action space $\mathcal{A}$), simplest exploration strategy is epsilon-greedy approach. During learning, a random action is selected by probability $\epsilon$ or greedy action (maximizing Q value) by $1-\epsilon$ as shown in \eqref{eqn:egreedy_policy}. 
%
\begin{equation}
\label{eqn:egreedy_policy}
\pi(a|s) = 
\begin{cases}
1-\epsilon,   & \text{if } a = \argmax_{a} Q(s, a)\\
\frac{\epsilon}{|\mathcal{A}|-1},     & \text{otherwise}
\end{cases} 
\end{equation}

Algorithm is summarized in Algorithm \ref{alg:dqn}. 

\begin{algorithm}[H]
	\SetAlgoLined
	\DontPrintSemicolon % Some LaTeX compilers require you to use \dontprintsemicolon instead
	Initialize: Replay memory $\mathcal{D}$ with capacity $N_{replay}$ \\
	Action value function parameters $\theta$ \\
	Target action value function parameters $\theta^- \leftarrow \theta$ \\
	Epsilon parameter for exploration $\epsilon$, Update delay parameter $d$ \\
	\For{$\text{episode} = 1, M $}{
		Recieve initial state $s_1$; \\
		\For{$t = 1, T$}{
			Select random action $a_t$ with probability $\epsilon$, otherwise greedy action $a_t = \argmax_{a} Q(s_t, a; \theta)$; \\
			Execute action $a_t$ and recieve reward $r_t$ and next state $s_{t+1}$; \\
			Store experience tuple $e_t = (s_t, a_t, r_t, s_{t+1})$ to $\mathcal{D}$ ; \\
			Sample random batch with $N$ transitions from $\mathcal{D}$ as $\mathcal{D}_{r}$; \\
			Set $Y_j^{DQN} = \begin{cases}
			r_j + \gamma \max_{a'} Q(s_{j+1},a';\theta^-) & \text{if } s_{j+1} \text{ not terminal } \\
			r_j & \text{if } s_{j+1} \text{ terminal }
			\end{cases}
			\forall e_j \in \mathcal{D}_{r}$;
			Update $\theta$ by minimizing $ \frac{1}{N}\sum_{e_j \in \mathcal{D}_{r}} \big( Y_j^{DQN} - Q(s_j,a_j;\theta) \big) ^ 2$ with a single optimization step; \\
			\lIf{$t \mod d$}{
				Update target network: $\theta^- \leftarrow \theta$;
			}
		}
	}
	\caption{Deep Q Learning with Experience Replay}
	\label{alg:dqn}
\end{algorithm}

\subsubsection{Double Deep Q Learning}

In DQN, max operator is used to select and evaluate action on the same network \ref{eqn:dqn_ntarget}. This yields overestimated value estimations in noisy environments. Therefore, action selection and value estimation is decoupled in target evaluation to overcome $Q$ function overestimation \cite{van_hasselt_deep_2015}.

\begin{equation}
\label{eqn:ddqn_ntarget}
Y_t^{DDQN} = r_t + \gamma Q(s_{t+1}, \argmax_{a'} Q(s_{t+1}, a'; \theta_i );\theta^-)
\end{equation}

Learning process is same with DQN except target value.

\subsection{Deterministic Actor Critic Learning}
In value based methods are not suitable for continious action spaces. Therefore, policy is explicitly defined instead of maximizing $Q$ function. Policy function can be either stochastic or deterministic.  Deterministic actor-critic is kind of learning which uses deterministic policy \cite{silver_deterministic_2014}. It can be thought as continious version of Q Learning. Value function is called critic while policy function is called actor.

In discrete action space, policy is naturally obtained by argmax operation on $Q$ function as in \ref{eqn:policy_deterministic_q}. In DPG, policy $\mu$ is parametrized by another set of parameters $\theta^\mu$ while value function $Q$ is parametrized by $\theta^Q$. 

The ultimate goal is to maximize value function $V$. It is done by selecting action which maximizes $Q$ value as policy \eqref{eqn:policy_deterministic_q}. Therefore, value function is the criteria to be maximized by solving parameters $\theta^\mu$ given $\theta^Q$. 
%
\begin{equation}
\label{eqn:dpg_value_maximization}
\theta^\mu = \argmax Q(s_t, \mu(s_t;\theta^\mu);\theta^Q)
\end{equation}

In order to learn policy, $Q$ function should also be learned simulatenously. For $Q$ function approximation, target value is parametrized by $\theta^Q$ and $\theta^\mu$ \eqref{eqn:dpg_target}. And this target is used to learn $Q$ function by minimizing least squares loss \eqref{eqn:dpq_loss}.
%
\begin{equation}
\label{eqn:dpg_target}
Y_t^{DPG} = r_t + \gamma Q(s_{t+1}, \mu(s_{t+1};\theta^\mu);\theta^Q)
\end{equation}
%
\begin{equation}
\label{eqn:dpq_loss}
\mathcal{L}_t(\theta^Q) = \big( Y_t^{DPG} - Q(s_t,a_t;\theta^Q) \big) ^ 2
\end{equation}

Note that both $\theta^Q$ and $\theta^\mu$ should be learned at the same time. Therefore, parameters are updated simulatenously during learning iterations.

\subsubsection{Deep Deterministic Policy Gradient}
Deep Deterministic Policy Gradient is continuous compliment of Deep Q learning using deterministic policy \cite{lillicrap_continuous_2019}. It uses experience replay and target networks.  

Similar to Deep Q learning, there are target networks parametrized by $\theta^{\mu^-}$ and $\theta^{Q^-}$ along with main networks parametrized by $\theta^{\mu}$ and $\theta^{Q}$. While target networks are updated in fixed number of steps in DQN, DDPG updates target network parameters at each step with polyak averaging as follows:
%
\begin{equation}
\label{eqn:target_update}
\theta^- \leftarrow \tau \theta + (1-\tau) \theta^-
\end{equation}

The $\tau$ is an hyperparameter indicating how slow the target network is updated and usually close to zero.

Policy network parameters are learned by maximizing resulting expected value \ref{eqn:ddpg_value_maximization}. Note that value network parameters are assumed to be learned.

\begin{equation}
\label{eqn:ddpg_value_maximization}
\theta^\mu = \argmax \mathbb{E}_{s \sim U(\mathcal{D})} \Big[ Q(s, \mu(s_t;\theta^\mu);\theta^Q) \Big]
\end{equation}
 
Target networks are used to predict targe value \ref{eqn:ddpg_target}. This target is used to learn $Q$ function by minimizing least squares loss \ref{eqn:ddpq_loss} in each iteration.

\begin{equation}
\label{eqn:ddpg_target}
Y_t^{DDPG} = r_t + \gamma Q(s_{t+1}, \mu(s_{t+1};\theta^{\mu^-});\theta^{Q^-})
\end{equation}

\begin{equation}
\label{eqn:ddpg_loss}
\mathcal{L}_i(\theta_i) = \mathbb{E}_{s,a,r,s'\sim U(\mathcal{D})}\Big[\big( Y^{DDPG} - Q(s,a;\theta^Q_i) \big) ^ 2 \Big]
\end{equation}

In DDPG, value and policy network parameters are learned simultaneously. During learning, exploration noise is added to each selection. In original paper \cite{lillicrap_continuous_2019}, authors proposed to use Ornstein Uhlenbeck Noise \cite{uhlenbeck_theory_1930} which has temporal correlation for efficiency. However, simple gaussian noise or another noise is also possible.

Algorithm is summarized in Algorithm \ref{alg:ddpg}. 

\begin{algorithm}[H]
	\SetAlgoLined
	\DontPrintSemicolon % Some LaTeX compilers require you to use \dontprintsemicolon instead
	Initialize: Replay memory $\mathcal{D}$ with capacity $N_{replay}$ \\
	Policy and action value function parameters $\theta^{\mu}$, $\theta^Q$ \\
	Target policy and action value function parameters $\theta^{\mu^-} \leftarrow \theta^{\mu}$, $\theta^{Q^-} \leftarrow \theta^{Q}$ \\
	Random process $\mathcal{N}$ as exploration noise\\
	\For{$\text{episode} = 1, M $}{
		Recieve initial state $s_1$; \\
		\For{$t = 1, T$}{
			Select action $a_t = \mu(s_t; \theta^{\mu}) + \epsilon$ where $\epsilon \sim \mathcal{N}$; \\
			Execute action $a_t$ and recieve reward $r_t$ and next state $s_{t+1}$; \\
			Store experience tuple $e_t = (s_t, a_t, r_t, s_{t+1})$ to $\mathcal{D}$ ; \\
			Sample random batch with $N$ transitions from $\mathcal{D}$ as $\mathcal{D}_{r}$; \\
			Set $Y_j^{DDPG} = \begin{cases}
			r_j + \gamma Q(s_{j+1},\mu(s_{j+1}; \theta^{\mu^-}); \theta^{Q^-}) & \text{if } s_{j+1} \text{ not terminal } \\
			r_j & \text{if } s_{j+1} \text{ terminal }
			\end{cases}
			\forall e_j \in \mathcal{D}_{r}$; \\
			Update $\theta^Q$ by minimizing $ \frac{1}{N}\sum_{e_j \in \mathcal{D}_{r}} \big( Y_j^{DDPG} - Q(s_j,a_j;\theta^Q) \big) ^ 2$ with a single optimization step; \\
			Update $\theta^{\mu}$ by maximizing $ \frac{1}{N}\sum_{e_j \in \mathcal{D}_{r}} Q(s_j,a_j;\theta^Q)$ with a single optimization step; \\
			Update target networks \\
			$\theta^{\mu^-} \leftarrow \tau \theta^{\mu} + (1-\tau) \theta^{\mu^-}$ \\
			$\theta^{Q^-} \leftarrow \tau \theta^{Q} + (1-\tau) \theta^{Q^-}$;
		}
	}
	\caption{Deep Deterministic Policy Gradient}
	\label{alg:ddpg}
\end{algorithm}

\subsubsection{Twin Delayed Deep Deterministic Policy Gradient}
Twin Delayed Deep Deterministic Policy Gradient \cite{fujimoto_addressing_2018} is improved version of DDPG with higher stability and efficiency. There are three main tricks.

\textbf{Target Policy Smoothing}: For target value assesing, actions are obtained from target policy network in DDPG, while a clipped zero centered gaussian noise is added to actions in TD3 \ref{eqn:td3_target_action}. This regularizes learning process by smoothing effects of actions on value. 

\begin{equation}
\label{eqn:td3_target_action}
a'(s') = \mu(s';\theta^{\mu^-}) + \text{clip}(\epsilon, -c, c), \quad \epsilon \sim \mathcal{N}(0, \sigma)
\end{equation}

\textbf{Clipped Double Q Learning}: There are two different $Q$ networks with their targets. During learning, both of them are learned from single target value. This value is assessed by using whichever of two networks give smaller of it. This allows to escape overestimation of values. 

\begin{equation}
\label{eqn:td3_target}
Y_t^{TD3} = r_t + \gamma \min_{k\in\{1,2\}} Q(s_{t+1}, ;a'(s_{t+1});\theta^{Q_k^-})
\end{equation}

Policy is learned by maximizing output of first value network.

\begin{equation}
\label{eqn:td3_value_maximization}
\theta^\mu = \argmax \mathbb{E}_{s \sim U(\mathcal{D})} \Big[ Q(s, \mu(s_t;\theta^\mu);\theta^{Q_1}) \Big]
\end{equation}

\textbf{Delayed Policy Updates}: During learning, policy network and target networks are updated less frequently (at each fixed number of step) than value network. Since policy network parameters are learned by maximizing value network, it should be learned slower.

Algorithm is summarized in Algorithm \ref{alg:td3}. 

\begin{algorithm}[H]
	\SetAlgoLined
	\DontPrintSemicolon % Some LaTeX compilers require you to use \dontprintsemicolon instead
	Initialize: Replay memory $\mathcal{D}$ with capacity $N_{replay}$ \\
	Policy and action value function parameters $\theta^{\mu}$, $\theta^Q_1$, $\theta^Q_2$  \\
	Target policy and action value function parameters $\theta^{\mu^-} \leftarrow \theta^{\mu}$, $\theta^{Q^-}_1 \leftarrow \theta^{Q}_1$, $\theta^{Q^-}_2 \leftarrow \theta^{Q}_2$ \\
	Random process $\mathcal{N}$ as exploration noise \\
	Target policy smoothing parameters $\sigma$, $c$, Update delay parameter $d$ \\
	\For{$\text{episode} = 1, M $}{
		Recieve initial state $s_1$; \\
		\For{$t = 1, T$}{
			Select action $a_t = \mu(s_t; \theta^{\mu}) + \epsilon$ where $\epsilon \sim \mathcal{N}$
			Execute action $a_t$ and recieve reward $r_t$ and next state $s_{t+1}$; \\
			Store experience tuple $e_t = (s_t, a_t, r_t, s_{t+1})$ to $\mathcal{D}$ ; \\
			Sample random batch with $N$ transitions from $\mathcal{D}$ as $\mathcal{D}_{r}$; \\
			Evaluate target actions for value target
			$a'(s_{j+1}) = \mu(s_{j+1};\theta^{\mu^-}) + \text{clip}(\epsilon, -c, c), \quad \epsilon \sim \mathcal{N}(0, \sigma) \quad \forall e_j \in \mathcal{D}_{r}$; \\
			$\forall k \in \{1,2\}$, set $Y_{jk}^{TD3} = \begin{cases}
			r_j + \gamma Q(s_{j+1},a'(s_{j+1}); \theta^{Q^-}_k) & \text{if } s_{j+1} \text{ not terminal } \\
			r_j & \text{if } s_{j+1} \text{ terminal }
			\end{cases}
			\forall e_j \in \mathcal{D}_{r} $; \\
			Set $Y_j^{TD3} = \min(Y_{j1}^{TD3}, Y_{j2}^{TD3})$; \\
			Update $\theta^Q_1$ $\theta^Q_2$ by seperately minimizing $ \frac{1}{N}\sum_{e_j \in \mathcal{D}_{r}} \big( Y_j^{TD3} - Q(s_j,a_j;\theta^Q_k) \big) ^ 2 \quad \forall k \in \{1,2\}$ with a single optimization step; \\
			\uIf{$t \mod d$}{
				Update $\theta^{\mu}$ by maximizing $ \frac{1}{N}\sum_{e_j \in \mathcal{D}_{r}} Q(s_j,a_j;\theta^Q_1)$ with a single optimization step; \\
				Update target networks \\
				$\theta^{\mu^-} \leftarrow \tau \theta^{\mu} + (1-\tau) \theta^{\mu^-}$ \\
				$\theta^{Q^-}_k \leftarrow \tau \theta^{Q}_k + (1-\tau) \theta^{Q^-}_k \quad \forall k \in \{1,2\}$ ;
			}
		}
	}
	\caption{Twin Delayed Deep Deterministic Policy Gradient}
	\label{alg:td3}
\end{algorithm}
\chapter{CONCLUSION AND FUTURE WORK}
\label{chap:conclusion}

\section{Conclusion}

In this thesis, bipedal robot walking is investigated by deep  reinforcement learning methods. 

Twin Delayed Deep Deterministic Policy Gradient (TD3) 

As stated in previous chapters, most of the real world environments are partially observable. In Bipedal-Walker-Hardcore, the environment is also partially observable since agent cannot observe behind of it lacks of acceleration sensors which is better to have for controlling mechanical systems. Therefore, we used Long Short Term Memory and Transformer Neural Networks to capture more information from past observations compared to single instant observation. Along with them, we also implemented Residually connected Feed Forward Neural Network using single instant observation. \\

We were expecting to have better results with xxyy.

\section{Future Work}


\section{Backpropagation}
Some optimization theory\\

\section{Building Units}
\label{sec:building_units}
\subsection{Perceptron}
Perceptron is a binary classifier model. In order to allocate input $x$ into a class, feature vector $\phi(x) \in \mathbb{R}^{1 \times d_k}$ is generated by a fixed nonlinear function. Then, a linear model is generated with linear transformation weights $W \in \mathbb{R}^{d_k \times 1} $ in the following form \eqref{eqn:perceptron1}. \\
\begin{equation}
\label{eqn:perceptron1}
y = f(\phi(x) W)
\end{equation}
where $f$ is called activation function. For perceptron, it is defined as step function \eqref{eqn:stepfun} while other functions like sigmoid, tanh can also be defined. \\
\begin{equation}
\label{eqn:stepfun}
f(a) = 
\begin{cases}
1,   & \text{if } a\geq 0\\
0,   & \text{otherwise}
\end{cases} 
\end{equation}
A learning algorithm of a perceptron aims determining the parameter vector $W$. It is best motivated by error minimization of data samples once a loss function is constructed. \\
\subsection{Activation Functions}
As in \eqref{eqn:stepfun}, step function is used in perceptron. However, any other nonlinear function can be used instead. This nonlinearity allows a model to capture nonlinearity in data. There are tons of activation function in use today. Commonly used activations are \textit{sigmoid}, \textit{hyperbolic tangent (Tanh)}, \textit{rectified linear unit (ReLU)}, \textit{gaussian error linear unit (GELU)}. \\
\textbf{Sigmoid Function}: Sigmoid ($\sigma$) is used when an output is required to be in $[0,1]$, like probability value. However, it has small derivative values at value near 0 and 1. \\
\begin{equation}
\label{eqn:sigmoid_fcn}
\sigma(x) = \frac{1}{1+e^{-x}}
\end{equation}
\textbf{Hyperbolic Tangent}: Tanh is used when an output is required to be in $[-1,1]$. It has similar behavior with sigmoid function except it is zero centered. Their difference is visualized in \figref{fig:sigmoid_tanh}. \\
\begin{equation}
\label{eqn:tanh_fcn}
\tanh(x) = \frac{e^x - e^{-x}}{e^x + e^{-x}}
\end{equation}
\textbf{ReLU}: ReLU is a simple function mapping negative values to zero while passing positive values as it is. It is computationally cheap and allows to train deep and complex networks \cite{glorot_deep_2011}. \\
\begin{equation}
\label{eqn:relu_fcn}
\textrm{ReLU}(x) = \max(0, x)
\end{equation}
\textbf{GELU}: GELU is smoothed version of ReLU function. It is continious but non-convex and has several advantages \cite{hendrycks_gaussian_2020}. ReLU and GELU are visualized in  \figref{fig:relu_gelu}. \\
\begin{equation}
\label{eqn:gelu_fcn}
\textrm{GELU}(x) = x \Phi(x) = \frac{x}{2} \bigg[ 1 + \textrm{erf} \Big( \frac{x}{\sqrt{2}} \Big) \bigg]
\end{equation}
\begin{figure}
	\begin{subfigure}{.5\textwidth}
		\centering
		\includegraphics[width=0.9\textwidth]{figures/ml_theory/relu_gelu.png}
		\caption{ReLU and GELU functions}
		\label{fig:relu_gelu}
	\end{subfigure}
	\begin{subfigure}{.5\textwidth}
		\centering
		\includegraphics[width=0.9\textwidth]{figures/ml_theory/sigmoid_tanh.png}
		\caption{Sigmoid and Tanh functions}
		\label{fig:sigmoid_tanh}
	\end{subfigure}
	\caption{Activation Functions}
	\label{fig:activation_functions}
\end{figure}
\subsection{Layer Normalization}
Layer normalization is a layer to overcome unstablity and divergence during learning \cite{ba_layer_2016}. Given an input $x \in \mathbb{R}^K$, mean and variance statistics are evaluated along the dimensions \eqref{eq:layernorm_statistics}. \\
\begin{equation}
\label{eq:layernorm_statistics}
\begin{split}
\mu = & \frac{1}{K} \sum_{n=1}^{K} x_k \\
\sigma^2 = & \frac{1}{K} \sum_{n=1}^{K} (x_k-\mu)^2
\end{split}
\end{equation} 
Then, the input is first scaled to have zero mean and unity variance along dimensions. The term $\epsilon$ is added to prevent division by zero. Optionally, the result is scaled by elementwise multiplication by $\gamma \in \mathbb{R}^K$ and addition by $\beta \in \mathbb{R}^K$ where these are learnable parameters. \\
\begin{equation}
\label{eqn:layernorm}
\mathrm{LN}(x) = \frac{x-\mu}{\sigma+\epsilon} * \gamma + \beta
\end{equation}

\section{Neural Network Types}
\label{sec:nnet_types}

\subsection{Feed Forward Neural Networks (Multilayer Perceptron)}
Structure of perceptron make a way for feed forward neural layers. Unlike stated below, a neural layer might output multiple values (say $o \in \mathbb{R}^{1 \times d_o}$) as vector from input (say $x \mathbb{R}^{1 \times d_x}$). Such a setting forces parameter $W \in \mathbb{R}^{d_x \times d_o} $ to be a matrix. Moreover, activation function is not necessariliy step function. It can be any nonlinear function like sigmoid, tanh, relu etc. Feed Forward Neural Networks are generalization of perceptron algorithm to approximate any function $f^*$. Neural layers are stacked to construct deep feed forward neural network. It defines a nonlinear mapping $y=f(x;\theta)$ between input $x$ and output $y$, parametrized by parameters $\theta = \{W\}_n,n=1,…,N.$.

Assuming input signal is $x$ (output of previous layer), activation value of the layer ($h$) is evaluated as by linear transformation followed by nonlinear activation $f$ \eqref{eqn:mlpact} applied elementwise.

\begin{equation}
\label{eqn:mlpact}
\text{net} = x W + b \quad \text{ and }\quad  h = f(\text{net})
\end{equation}

\subsection{Recurrent Neural Networks}
Recurrent Neural Networks (RNNs) \cite{rumelhart_learning_1986} are type of neural network to process sequential data. It is specialized for data having sequential topology. It is used commonly used for sequence based applications.

Sequential data can be inferred by Recurrent Neural Networks. In Feed Forward Layers, output only depends on its input, while Recurrent Layer output is dependent on both input at time $t$ and its output in previous time step $t-1$.

RNN can be thought as multiple copies of same network which passes message to its successor through time. A RNN layer is similar to MLP layer \ref{eqn:mlpact}, except input is concatenation of output feedback and input itself \ref{eqn:rnnact}.

Given input sequence $x \in \mathbb{R}^{T \times d_x}$, output sequence $h \in \mathbb{R}^{T \times d_h}$ is evaluated recursively. Initial output $h_0$ can be either parametrized or assigned as zeros vector. Again, nonlinear activation $f$ \ref{eqn:mlpact} applied element-wise. 

\begin{equation}
\label{eqn:rnnact}
net_t = h_{t-1} \tilde{W} + x_t W + b \quad \text{ and }\quad  h_t = f(net_t)
\end{equation}


\begin{figure}
	\centering
	\includegraphics[width=0.5\textwidth]{figures/ml_theory/rnn_vs_ffnn_layer.png}
	\caption{Recurrent Layer (left) and Feed Forward Layer (right) illustration.}
	\label{fig:rnn_vs_ffnn}
\end{figure}

\subsubsection{Long Term Dependence Problem of Vanilla RNNs}


Conventional RNNs have problem with vanishing/exploding gradient problem.  As the sequence gets longer, effect of initial inputs in sequence decreases. This causes long term dependence problem. If information from initial inputs required, gradients either vanish or explode. 

In order to overcome this problem another architecture is developed called Long Short Term Memory (LSTM) \cite{hochreiter_long_1997}.

\subsubsection{Long Short Term Memory}
LSTM is a special type of RNN. It is explicitly designed to allow learning long-term dependencies. A single LSTM cell has 4 neural layer while vanilla RNN layer has only one neural layer. In addition to hidden state $h_t$ , there is another state called cell state $C_t$. Information flow is controlled by 3 gates. 

\begin{figure}
	\centering
	\includegraphics[width=0.95\textwidth]{figures/ml_theory/lstm/lstm_module.png}
	\caption{LSTM Cell.}
	\label{fig:lstm_cell}
\end{figure}

\textbf{Forget Gate}: Forget gate controls past memory. According to input, past memory is either kept or forgotten. Sigmoid function ($\sigma$) is used as activation function, applied elementwise.

\begin{equation}
\label{eqn:lstm_forget}
f_t = \sigma( [h_{t-1}; x_t] W_f + b_f) 
\end{equation}

\textbf{Input Gate}: Input gate controls contribution from input to cell state (memory). Hyperbolic tangent layer creates new candidate of cell state from input.

\begin{equation}
\label{eqn:lstm_inp}
i_t = \sigma( [h_{t-1}; x_t] W_i + b_{i}) 
\end{equation}

\begin{equation}
\label{eqn:lstm_cellstcand}
\hat{C}_t = \tanh( [h_{t-1}; x_t] W_C + b_C) 
\end{equation}

\textbf{Cell State Update}: Once what are to be forget and added are decided, cell state is updated.

\begin{equation}
\label{eqn:lstm_cellstupt}
C_t = f_t \odot C_{t-1} + i_t \odot \hat{C}_t
\end{equation}


\textbf{Output Gate}: Sigmoid layer decides what part of new cell state to be output. Cell state is filtered by tanh to push values to be in $(-1,1)$.

\begin{equation}
\label{eqn:lstm_out}
o_t = \sigma( [h_{t-1}; x_t] W_o + b_o) 
\end{equation}

\begin{equation}
h_t = o_t \odot \tanh(C_t)
\end{equation}

\subsection{Attention Mechanism}
As stated earlier, recurrent neural networks are prone to forget long term dependencies. LSTM and GRU are invented to overcome this problem. Although they reduced this problem, they cannot attend specific parts of the input. For example, for sentiment analysis, specific keywords are important to determine sentiment of a sentence. However, last state of encoded input is not able to remember that words. Therefore, people came with the idea of weighted avearing all states through time where weights depends on both input and output. 

Assume that input sequence $X \in \mathbb{R}^{T \times d_X}$ is encoded to $H \in \mathbb{R}^{T \times d_H}$. The context vector is calculated using weight vector $\alpha(t)$. 

Calculation of weight vector depends on the task. For each time step, a score function is calculated between hidden state $H \in \mathbb{R}^{T \times d_H}$ and query $\boldsymbol{q}$ (which may be many things depending on task). Also, score function is also depends on choice. Then, attention score is $\boldsymbol{\alpha} \in \mathbb{R}^{T}$ is calculated using arbitrary function $f$ depending on choice.

\begin{equation}
\begin{split}
\boldsymbol{\alpha} = & f(\boldsymbol{q}, H) \\
\mathrm{Attention}(q, H) = & \sum_{\tau=0}^{T} \alpha_{\tau} h_{\tau}
\end{split}
\end{equation}


\subsubsection{Transformer}
The Transformer was proposed in the paper Attention is All You Need \cite{vaswani_attention_2017} . Unlike recurrent networks, this architecture is solely builded on attention layers. 

A transformer layer consists of feed-forward and attention layers, which makes the mechanism special. Like RNNs, it can be used as both encoder and decoder. While encoder layers attend to itself, decoder layers attends both itself and encoded input.

\textbf{Attention Layer}: An attention layer is a mapping from 3 vectors called query $Q \in \mathbb{R}^{T \times d_k}$, key $K \in \mathbb{R}^{T \times d_k}$ and value $V \in \mathbb{R}^{T \times d_v}$ to output, where $T$ is time length, $d_k$ and $d_v$ are embedding dimensions. Output is weighted sum of values $V$ while weights are evaluated by compatibility metric of query $Q$ and key $K$. In vanilla transformer, compatibility of query and key is evaluated by dot product, normalizing by $sqrt(d_k)$. For a query, dot product with all keys are evaluated, then softmax function is applied to get weights of values. This approach is called Scaled Dot-product Attention.

\begin{equation}
\mathrm{Attention}(Q, K, V) = \mathrm{softmax}(\frac{Q^{T} K}{\sqrt{d_k}}) V
\end{equation}

\textbf{Multi-Head Attention}: Instead of performing single attention; keys, queries and values are linearly projected from $d_m$ dimensional vector space to $h$ different spaces using projection matrices. Then, attention is done $h$ times, and results are then concatenated and linearly projected to final values of the layer.

Projection matrices are model parameters, $W^Q_i \in \mathbb{R}^{d_m \times d_k}$, $W^K_i \in \mathbb{R}^{d_m \times d_k}$, $W^V_i \in \mathbb{R}^{d_m \times d_v}$ for $i=1,...,h$. Also output matrix is used to project multiple values into single one, $W^O \in \mathbb{R}^{h d_v \times d_m}$.

\begin{equation}
\begin{split}
% It is better to write these as \DeclareMathOp
\mathrm{MHA}(Q,K,V) &=  \text{Concat}(a_1, a_2, ... a_h)W^O \\
a_i &=  \text{Attention}(QW^Q_i,KW^K_i,VW^V_i)
\end{split}
\end{equation}

\textbf{Feed Forward Layer}: Both encoder and decoder contains feed forward layer, containing two linear transformations with ReLU activation.

\begin{equation}
\mathrm{FFN}(x) = \text{ReLU}(xW_1+b)W_2+b_2
\end{equation}



\textbf{Encoder Layer}: Encoder Layer starts with a residual self attention layer. Self attention means that query, key and value are same vectors. Then it is followed by feed forward neural layer, Both sublayers are employed with resudial connection with layer normalization, i.e summation of layer input and output is passed through layer normalization.

\begin{equation}
\begin{split}
att = & \mathrm{LN}(x+ \mathrm{MHA}(x,x,x)) \\
y = & \mathrm{LN}(att+ \mathrm{FFN}(att))
\end{split}
\end{equation}

\textbf{Decoder Layer}: Similar to encoder layer, decoder layer has also self-attention and feed forward layers. In addition, there is another attention layer which is over encoder outputs. Same as encoder, all sublayers have resudial connection with layer normalization. Let's call encoded sequence $e \in \mathbb{R}^{T \times d_m}$ and decoded sequence $d \in \mathbb{R}^{T \times d_m}$ (masked). Assume that the sequence decoded up to $t$th point in sequence. Then, $d_{t+1}$ is calculated as follows.

\begin{equation}
\begin{split}
att = & \mathrm{LN}(d_{1:t}+\mathrm{MHA}(d_{1:t},d_{1:t},d_{1:t})) \\
dec = & \mathrm{LN}(att+ \mathrm{MHA}(att,e,e)) \\
d_{t+1} = & \mathrm{LN}(dec+ \mathrm{FFN}(dec))
\end{split}
\end{equation}

\textbf{Positional encoding}: Since there are no recurrent or convolutional architecture in the model, sequential information needs to be embedded. Positional encodings has same dimension, so that input embeddings can be added to at the beginning of encoder or decoder stacks. For position $pos$, $2i$ or $2i+1$th dimension has following values ($i \in \mathbb{N}$), as proposed in the original paper.

\begin{equation}
\begin{split}
\mathrm{PE}_{pos,2i} = \sin(pos/10000^{2i/d_m}) \\
\mathrm{PE}_{pos,2i+1} = \cos(pos/10000^{2i/d_m})
\end{split}
\end{equation}

\subsubsection{Pre-Layer Normalized Transformer}
Original transformer architecture includes layer normalization operations after attention and feed-forward layers. It is unstable since gradients of output layers are high, so Pre-Layer Normalized transformer is proposed by \cite{xiong_layer_2020}. Moreover,  Parisotto et al. Xiong et al. \cite{parisotto_stabilizing_2019}. proposed gated transformer which also includes layer normalizations before attention and feedforward layer. They also stated that although gated architecture improves many RL tasks drastically, non-gated pre-layer normalized transformer are good enough. 

\begin{figure}
	\centering
	\includegraphics[width=0.55\textwidth]{figures/ml_theory/post_pre_trsf.png}
	\caption{(a) Post-LN Transformer layer, (b) Pre-LN Transformer
		layer.}
	\label{fig:post_pre_trsf}
\end{figure}

Encoder equations are as follows.

\begin{equation}
\begin{split}
att = & x+ \mathrm{MHA}(\mathrm{LN}(x),\mathrm{LN}(x),\mathrm{LN}(x)) \\
y = & att+ \mathrm{FFN}(\mathrm{LN}(att))
\end{split}
\end{equation}

%Decoder equations are as follows.

%\begin{equation}
%\begin{split}
%att = & %d_{1:t}+\mathrm{MHA}(\mathrm{LN}(d_{1:t}),\mathrm{LN}(d_{1:t}),\mathrm%{LN}(d_{1:t})) \\
%dec = & att+ \mathrm{MHA}(\mathrm{LN}(att),\mathrm{LN}(att),\mathrm{LN}(att)) \\
%d_{t+1} = & dec+ \mathrm{FFN}(\mathrm{LN}(dec))
%\end{split}
%\end{equation}


\chapter{CONCLUSION AND FUTURE WORK}
\label{chap:conclusion}

\section{Conclusion}

In this thesis, bipedal robot walking is investigated by deep  reinforcement learning methods. 

Twin Delayed Deep Deterministic Policy Gradient (TD3) 

As stated in previous chapters, most of the real world environments are partially observable. In Bipedal-Walker-Hardcore, the environment is also partially observable since agent cannot observe behind of it lacks of acceleration sensors which is better to have for controlling mechanical systems. Therefore, we used Long Short Term Memory and Transformer Neural Networks to capture more information from past observations compared to single instant observation. Along with them, we also implemented Residually connected Feed Forward Neural Network using single instant observation. \\

We were expecting to have better results with xxyy.

\section{Future Work}


\section{Proposed Neural Networks}
\label{sec:proposed_networks}

For all networks, varing backbones used to encode state information from observations for both actor and critic networks. 
In critic network, actions are concatenated by state information coming from backbones. 
Then, this concatenated vector is passed through feed forward layer with GELU activation then a linear layer with single output. 
Before feeding observations to backbone, they are passed through a layer with layer normalization with tanh activation to 96 dimensional output. 
In actor network, backbone is followed by a single layer with tanh activation for action estimation. 
As backbones, following networks are proposed. 
Again, observations are passed through a layer with layer normalization with tanh activation to 96 dimensional output before feeding to backbone. 
Critic and Actor networks are illustrated in \figref{fig:nets} 

\begin{figure}
	\begin{subfigure}{.5\textwidth}
		\centering
		\includegraphics[width=0.97\linewidth]{figures/nets/critic.png}
		\caption{Critic Architecture}
		\label{fig:critic_net}
	\end{subfigure}
	\begin{subfigure}{.5\textwidth}
		\centering
		\includegraphics[width=0.55\linewidth]{figures/nets/actor.png}
		\caption{Actor Architecture}
		\label{fig:actor_net}
	\end{subfigure}
	\caption{Neural Architecture Design}
	\label{fig:nets}
\end{figure}

\subsection{Residual Feed Forward Network}

Incoming vector is passed through 2 layers with 192 dimensional hidden size and 96 dimensional output, where there is GELU activation between 2 layers. 
This output is summed with initial vector and this is lastly passed through layer normalization. 

\subsection{Long Short Term Memory}

Sequence of incoming vectors is passed though vanilla LSTM layer with 96 dimensional hidden state. 
Output at last time step is outputted. 

\subsection{Transformer (Pre-layer Normalized)}

Sequence of incoming vectors is passed though pre-layer normalized transformer with 192 dimensional feed forward layer with GELU activation. 
The output is lastly passed through layer normalization. 
During multi-head attention, only last state is fed as query so that attentions are calculated for only last state. 


\section{RL Method and Hyperparameters}
\label{sec:rlmethod}

TD3 and SAC is used as RL algorithm. 
Hyperparameters are selected by grid search and best performing values are used. Adam optimizer is used for optimization. 

In TD3, as exploration noise, Ornstein-Uhlenbeck noise is used, and standart deviation is multiplied  by $0.999$ at the end of each episode. 

All hyperparameters are found after a trial-error process. They  are summarized in \tabref{table:hyperparams_td3} and \tabref{table:hyperparams_sac}. Unlisted ones have default values which PyTorch gives. 

\begin{table}
	\begin{tabular}{|l||*{3}{c|}}\hline
		\backslashbox{Hyperparameter}{Model}
		&\makebox[5em]{RFFNN}&\makebox[5em]{LSTM}&\makebox[5em]{Transformer}\\\hline\hline
		$\eta$ (Learning Rate) & $1.0\times10^{-3}$ & $7.0\times10^{-4}$ & $1.0\times10^{-3}$\\\hline
		$\beta$ (Momentum) & \multicolumn{3}{|c|}{$(0.9, 0.999)$}\\\hline
		$\gamma$ (Discount Factor) & \multicolumn{3}{|c|}{$0.98$} \\\hline
		$N_{replay}$ (Replay Buffer Size) &\multicolumn{3}{|c|}{$500000$} \\\hline
		$N$ (Batch Size) &\multicolumn{3}{|c|}{$128$}\\\hline
		$d$ (Policy Delay) &\multicolumn{3}{|c|}{$2$}\\\hline
		$\sigma$ (Policy smoothing std) &\multicolumn{3}{|c|}{$0.2$}\\\hline
		$\tau$ (Polyak parameter) &\multicolumn{3}{|c|}{$0.005$}\\\hline
		Exploration &\multicolumn{3}{|c|}{$OU(\theta=4.0, \sigma=1.0)$}\\\hline
	\end{tabular}
	\caption{Hyperparmeters and Exploration of Learning Processes for TD3}
	\label{table:hyperparams_td3}
\end{table}
\noindent

\begin{table}
	\begin{tabular}{|l||*{3}{c|}}\hline
		\backslashbox{Hyperparameter}{Model}
		&\makebox[5em]{RFFNN}&\makebox[5em]{LSTM}&\makebox[5em]{Transformer}\\\hline\hline
		$\eta$ (Learning Rate) & $1.0\times10^{-3}$ & $1.0\times10^{-3}$ & $1.0\times10^{-3}$\\\hline
		$\beta$ (Momentum) & \multicolumn{3}{|c|}{$(0.9, 0.999)$}\\\hline
		$\gamma$ (Discount Factor) & \multicolumn{3}{|c|}{$0.98$} \\\hline
		$N_{replay}$ (Replay Buffer Size) &\multicolumn{3}{|c|}{$500000$} \\\hline
		$N$ (Batch Size) &\multicolumn{3}{|c|}{$128$}\\\hline
		$\tau$ (Polyak parameter) &\multicolumn{3}{|c|}{$0.005$}\\\hline
		$\alpha$ (Entropy Temperature) &\multicolumn{3}{|c|}{$0.01$}\\\hline
	\end{tabular}
	\caption{Hyperparmeters and Exploration of Learning Processes for SAC}
	\label{table:hyperparams_sac}
\end{table}
\noindent

\section{Results}

For each episode, episode scores are calculated by summing up rewards of each time step. 
In \figref{fig:td3_scatter_ep_rewards} and \figref{fig:sac_scatter_ep_rewards}, a scatter plot is visualized for each model's episode scores. 
In \figref{fig:td3_std_ep_rewards} and \figref{fig:sac_std_ep_rewards}, moving average and standard deviation is visualized for each model's episode scores. 

\begin{figure}
	\centering
	\includegraphics[width=0.95\textwidth]{figures/bipedal/SCATTER_TD3_RFFNN_LSTM_TRSF.png}
	\caption{Scatter Plot with Moving Average for Episode Scores (TD3)}
	\label{fig:td3_scatter_ep_rewards}
\end{figure}
\begin{figure}
	\centering
	\includegraphics[width=0.95\textwidth]{figures/bipedal/STD_TD3_RFFNN_LSTM_TRSF.png}
	\caption{Moving Average and Standard Deviation for Episode Scores (TD3)}
	\label{fig:td3_std_ep_rewards}
\end{figure} 

\begin{figure}
	\centering
	\includegraphics[width=0.95\textwidth]{figures/bipedal/SCATTER_SAC_RFFNN_LSTM_TRSF.png}
	\caption{Scatter Plot with Moving Average for Episode Scores (SAC)}
	\label{fig:sac_scatter_ep_rewards}
\end{figure}
\begin{figure}
	\centering
	\includegraphics[width=0.95\textwidth]{figures/bipedal/STD_SAC_RFFNN_LSTM_TRSF.png}
	\caption{Moving Average and Standard Deviation for Episode Scores (SAC)}
	\label{fig:sac_std_ep_rewards}
\end{figure} 

First of all, none of our approaches solved the problem since 300 points required in 100 random simulations as solution. 
However, our methods partially solved problems by exceeding 200 point limit, while some simulations yield around 280 points in all models. 

RFFNN seems to be enough for solving the problem, although there exist partial observability in the environment. 
That model reaches around 220 points in average with TD3 and 225 points with SAC. In addition, learning was unstable. 

The robot was able to walk by LSTM model but yield worse results and cannot exceed 120 points in average with both models. 

Transformer model yield best results by reaching 230 points with TD3 and exceeds 250 points with SAC. 

As Transformer and RFFNN are relatively succesfull, their behavior is visualized in \figref{fig:rffnn_simulation} and \figref{fig:trsf_simulation} for TD3 model. 
The main behavior difference is when the agent faces with a big hurdle. 
First model passes it by jumping while other does by taking a very big step, shown in \figref{fig:anim_rffnn_hurdle} and \figref{fig:anim_trsf_hurdle}.

In SAC simulations at best models, RFFNN and Transformer models do not exhibit a noteworthy difference.

Also, SAC model performed better than TD3 in general. 
Agent cannot exceed 280 points in any simulation with TD3 (\figref{fig:td3_scatter_ep_rewards}) but it exceeds 300 points with SAC (\figref{fig:sac_scatter_ep_rewards}). Also, moving average points is also higher when SAC policy is used. 

\begin{figure}
	\centering
	\begin{subfigure}{.9\textwidth}
		\centering
		\includegraphics[width=0.99\textwidth]{figures/bipedal/anim/ff_flat.png}
		\caption{Flat Surface}
		\label{fig:anim_rffnn_flat}
	\end{subfigure}
	\begin{subfigure}{.9\textwidth}
		\centering
		\includegraphics[width=0.99\textwidth]{figures/bipedal/anim/ff_stairs.png}
		\caption{Stairs}
		\label{fig:anim_rffnn_stairs}
	\end{subfigure}
	\begin{subfigure}{.9\textwidth}
		\centering
		\includegraphics[width=0.99\textwidth]{figures/bipedal/anim/ff_hurdle.png}
		\caption{Hurdle}
		\label{fig:anim_rffnn_hurdle}
	\end{subfigure}
	\begin{subfigure}{.9\textwidth}
		\centering
		\includegraphics[width=0.99\textwidth]{figures/bipedal/anim/ff_pitfall.png}
		\caption{Pitfall}
		\label{fig:anim_rffnn_pitfall}
	\end{subfigure}
	\caption{Walking Simulation of RFFNN model at best version with TD3}
	\label{fig:rffnn_simulation}
\end{figure}

\begin{figure}
	\centering
	\begin{subfigure}{.9\textwidth}
		\centering
		\includegraphics[width=0.99\textwidth]{figures/bipedal/anim/trsf_flat.png}
		\caption{Flat Surface}
		\label{fig:anim_trsf_flat}
	\end{subfigure}
	\begin{subfigure}{.9\textwidth}
		\centering
		\includegraphics[width=0.99\textwidth]{figures/bipedal/anim/trsf_stairs.png}
		\caption{Stairs}
		\label{fig:anim_trsf_stairs}
	\end{subfigure}
	\begin{subfigure}{.9\textwidth}
		\centering
		\includegraphics[width=0.99\textwidth]{figures/bipedal/anim/trsf_hurdle.png}
		\caption{Hurdle}
		\label{fig:anim_trsf_hurdle}
	\end{subfigure}
	\begin{subfigure}{.9\textwidth}
		\centering
		\includegraphics[width=0.99\textwidth]{figures/bipedal/anim/trsf_pitfall.png}
		\caption{Pitfall}
		\label{fig:anim_trsf_pitfall}
	\end{subfigure}
	\caption{Walking Simulation of Transformer model at best version with TD3}
	\label{fig:trsf_simulation}
\end{figure}

\section{Discussion}
These results are not enough to conclude on a superior neural network for all RL problems, because there are other factors such as DRL algorithm, number of episodes, network size etc. 
However, networks are designed to have similar sizes and good model requires to converge in less episodes. 
As a result, LSTM is superior to Transformer for our environment. 
In addition, it is possible to conclude that Transformers can be an option for partially observed RL problems.
Note that this is valid where layer normalization is applied before multihead attention and feed-forward layers \cite{xiong_layer_2020} as opposed to vanilla transformer proposed in \cite{vaswani_attention_2017}. 

Another result is that incorprating past observations did improve performance significantly since environment is partially observable.
Especially increasing history length keeps increasing performance for both LSTM and Transformer models. 
To address this issue better, we designed Transformer and RFFNN networks to be almost same once multi-head attention layer is removed. 
We observed significant performance gains as observation history is used as input to controller. 

The environment is a difficult one. 
There are really few available models with solution~\cite{noauthor_gymleaderboard_2021}. 
Apart from neural networks, there are other factors affecting performance such as RL algorithm, rewarding, exploration etc. 
In this work, all of them are adjusted such that the environment becomes solvable. 
Time frequency is reduced for sample efficiency and speed. 
Also, the agent is not informed for the terminal state when it reaches time limit. 
Lastly, punishment of falling reduced, so the agent is allowed to learn by mistakes. 
Those modifications are probably another source of our high performance. 

As RL algorithm, TD3 is selected first, since it is suitable for continuous RL. 
Ornstein-Uhlenbeck noise is used for better exploration since it has momentum, and variance is reduced by time to make agent learn small actions well in later episodes. 
However, we cannot feed 12 observation history to sequential models since TD3 failed to learn in that case.
In addition, SAC is used for learning along with TD3. 
Results are better compared to those of TD3. 
SAC policy maximizes randomness (entropy) if agent cannot get sufficient reward and this allows the agent to decide where/when to explore more or less. 
This way, SAC handles the sparse rewards from the environment better than TD3. 

\chapter{CONCLUSION AND FUTURE WORK}
\label{chap:conclusion_future_work}

\section{Conclusion}
\label{sec:conclusion}
For robot control by RL in real world, simulation is an important step. 
Usually, models are pretrained in simulation environment before learning in reality due to safety and exploration reasons. 
Today, RL is rarely used in real world applications due to safety and sample inefficiency problems. 

In this thesis, bipedal robot walking is investigated by deep  reinforcement learning due to complex dynamics in OpenAI Gym's simulation environment. 
TD3 and SAC algorthims are used since they are robust and well suited for continuous control. 
Also, environment is slightly modified by reward shaping, halving simulation frequency, cancelling terminal state information at time limit so that learning becomes easier.
  
As stated in previous chapters, most of the real world environments are partially observable. 
In BipedalWalker-Hardcore, the environment is also partially observable since agent cannot observe behind and it lacks of acceleration sensors, which is better to have for controlling mechanical systems. 
Therefore, we propose to use Long Short Term Memory and Transformer Neural Networks to capture more information from past observations unlike Residual Feed Forward Neural Network (RFFNN) using a single instant observation as input. 

RFFNN model performed well thanks to carefully selected hyperparameters and modifications on the environment. 
Since there is no major differences between sequential models (Transformer), the problem seems mainly about the rewards and exploration, not the partial observability. 

Although LSTM is used commonly for partially observed problems, results are the worst in this problem compared to other two models. Nonetheless, it keeps learning slowly. 
Although we cannot exactly detect what the reason is, it is probably due to functional properties of LSTM. 

Transformer model worked with the best performance among all models we applied in this work. 
It is surprising because it is not succesfully used in RL problems in general. 
In natural language processing, this type of attention models completely replace recurrent models recently, and our results seems promising for this in RL domain. 

The environment was difficult for exploration. SAC performed better than TD3 in handling exploration problem. SAC learns randomness during learning iterations. 



%
% References in Bibtex format goes into below indicated file with .bib extension
\bibliography{myBiblio} % filename: myBiblio.bib
% You can use full name of authors, 
% however most likely some of the Bibtex entries you will find, 
% will use abbreviated first names.
% If you don't want to correct each of them by hand, 
% you can use abbreviated style for all of the references
%\bibliographystyle{abbrv}
% However, IAM suggests to use
\bibliographystyle{iamBiblioStyle} % better than to use {plain or abbrv}
%%% APPENDIXES
\appendix
%
% input your appendix
% If you are not using minted style, then comment the first appendix below
% otherwise uncomment.
%uncomment%  \chapter{Proof of Some Theorem}
\label{app:mintedCodes}

This is appendix text.

\definecolor{myBgColour}{rgb}{0.99,0.99,0.99} % almost white

\setminted[python]{frame=single,
framesep=2mm,
baselinestretch=1.1,
bgcolor=myBgColour,
fontsize=\footnotesize,
linenos, autogobble,
python3=true}

\setminted[matlab]{frame=single,
framesep=2mm,
baselinestretch=1.1,
bgcolor=myBgColour,
fontsize=\footnotesize,
linenos, autogobble,
python3=true}

%\captionsetup[Listing]{format=plain,font={small},labelfont={bf}, aboveskip=-5px}
%\renewcommand{\theListing}{{\arabic{Listing}}}
%\setcounter{Listing}{0}


However, we place a python code here with a listing environment
Listing~\ref{lst:first}.	

\begin{listing}	
\begin{minted}{python}
# Python program to check if the input number is prime or not

num = 407

# take input from the user
# num = int(input("Enter a number: "))

# prime numbers are greater than 1
if num > 1:
   # check for factors
   for i in range(2,num):
       if (num % i) == 0:
           print(num,"is not a prime number")
           print(i,"times",num//i,"is",num)
           break
   else:
       print(num,"is a prime number")
       
# if input number is less than
# or equal to 1, it is not prime
else:
   print(num,"is not a prime number")
\end{minted}
\caption{This is the caption of this Listing environment\label{lst:first}}
\end{listing}

Also we wish to insert a MATLAB code, too.

\definecolor{myBackgroundColour}{rgb}{0.9,0.9,0.9} % almost gray
\begin{minted}[
frame=lines,
framesep=2mm,
baselinestretch=1.2,
bgcolor=myBackgroundColour,
fontsize=\footnotesize,
linenos
]{matlab}
function result = myprime(n)
% MATLAB program to check if the input number is prime or not

%% initially set output flag to true
 result = true;
%% iterate over all positive integers 2,3,...,n-1
%% if n is not divisible by any of these factors....it is prime
 if (n == 1)
     result = 'false';
 elseif (n == 2)
     result = 'true';
 else 
    for i=2:n-1,
        if (mod(n,i)==0)
           result = 'false';
        end
    end
 end
%% return "true" or "false" instead of 1 or 0  
 if (result)
    result = 'true';
 else
    result = 'false';
 end
\end{minted}


Furthermore, here are two files (myPythonCode.py and myMatlabCode.m) included.

\inputminted[
frame=single,
framesep=2mm,
baselinestretch=1.2,
bgcolor=myBackgroundColour,
fontsize=\footnotesize,
linenos
]{python}{myPythonCode.py}

\definecolor{myRed}{rgb}{0.95,0.1,0.1}

\begin{listing}
\inputminted[frame=single, linenos, bgcolor=myRed]{matlab}{myMatlabCode.m}
\caption{Here is the caption again}
\end{listing} % includes minted package examples!
%\chapter{Proof of Some Theorem}
\label{app:somethms}

This is appendix text.

\begin{listing}
  %\VerbListingBoxed{myMatlabCode.m}
  \VerbatimInput{myMatlabCode.m}
	%\inputminted{matlab}{myMatlabCode.m} % only if minted is used!
  %\VerbListing{myMatlabCode.m}
\caption{The \texttt{lintest} function in a floating ``listing'' environment.}
\label{mfile:linetest-3}
\end{listing}


%
%
% If you are a Ph.D. Student you need to insert a CV at the end of you thesis
% Check vita.tex for a simple CV template in Latex
%\input{vita.tex}
\end{document}
