\section{Problem Statement: Bipedal Walker Robot Control}
\label{sec:problem_statement}

In this work, we attempted to solve \textit{BipedalWalkerHardcore-v3}~\cite{noauthor_bipedalwalkerhardcore-v2_2021} from OpenAI's open source Gym~\cite{brockman_openai_2016} library using deep learning framework PyTorch~~\cite{paszke_pytorch_2019}.

\subsection{OpenAI Gym and Bipedal Walker Environment}
\label{ssec:gym_bipedal}

OpenAI Gym~\cite{brockman_openai_2016} is an open source framework, 
containing many environments to service the development of 
reinforcement learning algorithms. 

\textit{BipedalWalker} environments~\cite{noauthor_bipedalwalker-v2_2021, noauthor_bipedalwalkerhardcore-v2_2021} are parts of Gym environment library. 
One of them is classical where the terrain is relatively smooth, while other is a hardcore version containing ladders, stumps and pitfalls in terrain. 
Robot has deterministic dynamics but terrain is randomly generated at the beginning of each episode.
Those environments have continuous action and observation space. 
For both settings, the task is to move the robot forward as much as possible. 
Snapshots for both environments are depicted in \figref{fig:bipedal_walkers}.
\begin{figure}
	\begin{subfigure}{.5\textwidth}
		\centering
		\includegraphics[width=0.9\linewidth]{figures/bipedal/classic.png}
		\caption{BipedalWalker-v3 Snapshot~\cite{noauthor_bipedalwalker-v2_2021}}
		\label{fig:bipedal_walker_classic}
	\end{subfigure}
	\begin{subfigure}{.5\textwidth}
		\centering
		\includegraphics[width=0.9\linewidth]{figures/bipedal/hardcore.png}
		\caption{BipedalWalkerHardcore-v3 Snapshot~\cite{noauthor_bipedalwalkerhardcore-v2_2021}}
		\label{fig:bipedal_walker_hardcore}
	\end{subfigure}
	\caption{Bipedal Walkers Snapshots}
	\label{fig:bipedal_walkers}
\end{figure}

Locomotion of the Bipedal Walker is a difficult control problem due to following reasons: 
\begin{description}
	\item[Nonlinearity] The dynamics are nonlinear, unstable and multimodal. 
	Dynamical behavior of robot changes for different situations 
	like ground contact, single leg contact and double leg contact.
	\item[Uncertainity] The terrain where the robot walks also varies. 
	Designing a controller for all types of terrain is difficult.
	\item[Reward Sparsity] Overcoming some obstacles requires a specific maneuver, which is hard to explore sometimes.	
	\item[Partially Observability] The robot observes 
	ahead with lidar measurements and cannot observe behind. 
	In addition, it lacks of acceleration sensors.
\end{description}

These reasons make it also hard to implement analytical methods for control tasks. 
However, DRL approach can easily overcome nonlinearity and uncertainity problems.

On the other hand, reward sparsity problem brings local minimums to objective function of optimal control. However, this can be challenged by a good exploration strategy and reward shaping. 

For the partial observability problem, more elegant solution is required. 
This is, in general, achieved by creating a belief state from the past observations to inform the agent. 
Agent uses this belief state to choose how to act. 
If the belief state is evaluated sufficiently, 
this increases performance of the agent.
Relying on instant observations is also possible, 
and this may be enough sometimes if advanced type of control is not required. 

\subsection{Deep Learning Library: PyTorch}
\label{dl_pytorch}
PyTorch is an open source library developed by Facebook's AI Research Lab (FAIR)~\cite{paszke_pytorch_2019}. 
It is based on Torch library~\cite{collobert_torch7_2011} and has Python and C++ interface. 
It is an automatic differentation library with accelerated mathematical operations backed by graphical processing units (GPUs). 
The clean pythonic syntax made it one of the most famous deep learning tool among researches. 
