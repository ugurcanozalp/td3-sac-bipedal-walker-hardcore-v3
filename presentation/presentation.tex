% Copyright 2004 by Till Tantau <tantau@users.sourceforge.net>.
%
% In principle, this file can be redistributed and/or modified under
% the terms of the GNU Public License, version 2.
%
% However, this file is supposed to be a template to be modified
% for your own needs. For this reason, if you use this file as a
% template and not specifically distribute it as part of a another
% package/program, I grant the extra permission to freely copy and
% modify this file as you see fit and even to delete this copyright
% notice.

\documentclass{beamer}

% There are many different themes available for Beamer. A comprehensive
% list with examples is given here:
% http://deic.uab.es/~iblanes/beamer_gallery/index_by_theme.html
% You can uncomment the themes below if you would like to use a different
% one:
%\usetheme{AnnArbor}
%\usetheme{Antibes}
%\usetheme{Bergen}
%\usetheme{Berkeley}
%\usetheme{Berlin}
%\usetheme{Boadilla}
%\usetheme{boxes}
%\usetheme{CambridgeUS}%%%%%%%%%%%%%%%%%%%%%%%%%%%%%%
%\usetheme{Copenhagen}
%\usetheme{Darmstadt}
%\usetheme{default}
%\usetheme{Frankfurt}
%\usetheme{Goettingen}
%\usetheme{Hannover}
%\usetheme{Ilmenau}
%\usetheme{JuanLesPins}
%\usetheme{Luebeck}
\usetheme{Madrid}%%%%%%%%%%%%%%%%%
%\usetheme{Malmoe}
%\usetheme{Marburg}
%\usetheme{Montpellier}
%\usetheme{PaloAlto}
%\usetheme{Pittsburgh}
%\usetheme{Rochester}
%\usetheme{Singapore}
%\usetheme{Szeged}
%\usetheme{Warsaw}
%%
%%
%
\setbeamertemplate{theorems}[numbered]
\setbeamertemplate{caption}[numbered]

%use package
\usepackage{multirow}
\usepackage{hyperref}
\usepackage{amssymb}
\usepackage{graphicx}
%\usepackage{algorithm}
\usepackage{amsmath,amsfonts}
\usepackage{amsthm}
\usepackage{amsmath}
\usepackage{xcolor}
\usepackage{dsfont}
\usepackage{epsfig}
\usepackage{color}
\usepackage{textcomp}
%\usepackage{beamerthemesplit}
%\usepackage{{ntheorem}}
%\usepackage[justification=centering]{caption}
\usepackage{subfigure}% Support for small, `sub' figures and tables
%\usepackage{subcaption}
\usepackage[utf8]{inputenc}
%\usepackage{tikz}
%\usepackage[tightpage]{preview}
%\usepackage{cite}
%\usepackage[scaled]{helvet}
%\usepackage[round]{natbib}
%%%<
%\usepackage{verbatim}
%%%%
%
%\logo{\includegraphics[height=1cm]{logo.png}\vspace{220pt}}
%\logo{\includegraphics[height=1cm]{logo.png}}

\title[Write a short title]{The Title of Your Presentation}

% A subtitle is optional and this may be deleted
%\subtitle{Optional Subtitle}

\author[Speaker]{F. author*\inst{1} \and S. author\inst{2} \and T. author\inst{3}}
% - Give the names in the same order as the appear in the paper.
% - Use the \inst{?} command only if the authors have different
%   affiliation.

\institute[METU] % (optional, but mostly needed)
{
  \inst{1}%
	Financial Mathematics
	\inst{2}%
		Actuarial Sciences
	\inst{3}
	Scientific Computing\\
	
	Institute of Applied Mathematics\\
  Middle East Technical University\\
	
 	}
% - Use the \inst command only if there are several affiliations.
% - Keep it simple, no one is interested in your street address.

\date[Conference Short name]{Conference long name\\
\today}
% - Either use conference name or its abbreviation.
% - Not really informative to the audience, more for people (including
%   yourself) who are reading the slides online

\subject{STATISTICS}
% This is only inserted into the PDF information catalog. Can be left
% out.

% If you have a file called "university-logo-filename.xxx", where xxx
% is a graphic format that can be processed by latex or pdflatex,
% resp., then you can add a logo as follows:

 \pgfdeclareimage[height=1cm]{university-logo}{logo.jpg}
 \logo{\pgfuseimage{university-logo}}

% Delete this, if you do not want the table of contents to pop up at
% the beginning of each subsection:
\AtBeginSubsection[]
{
  \begin{frame}<beamer>{Outline}
    \tableofcontents[currentsection,currentsubsection]
  \end{frame}
}

% Let's get started
\begin{document}

\begin{frame}
  \titlepage
\end{frame}

\begin{frame}{Outline}
  \tableofcontents
  % You might wish to add the option [pausesections]
\end{frame}

% Section and subsections will appear in the presentation overview
% and table of contents.

\section{Introduction}
\begin{frame}
You can write your motivation and aim in this part. You may add more frame if it is necessary.
\end{frame}
\subsection{Citation}
\begin{frame}
You can use your bib-file for citations as like this example\cite{Chauvet2016}.
\end{frame}
%%%%%%%%%%%%%%%%%%%%
%%%%%%%%%%%%%%%%%%%%
\section{Figure}
\subsection{Only 1 Figure}
\begin{frame}
\begin{figure}
    \centering
    \includegraphics[width = 0.85\textwidth]{logo.jpg}
    \caption{Enjoy your life}
  \end{figure}
\end{frame}
%%%%%%%%%%%%%%%%%
%%%%%%%%%%%%%%%%%%%
\subsection{Multiple figure}
\begin{frame}
\begin{columns}[onlytextwidth]
\begin{column}{.45\textwidth}
\begin{figure}
  \includegraphics[width=\textwidth]{logo.jpg}
  \caption{Don't surprise}
\end{figure}
\end{column}
\hfill
\begin{column}{.45\textwidth}
\begin{figure}
  \includegraphics[width=\textwidth]{logo.jpg}
  \caption{You can do}
\end{figure}
\end{column}
\end{columns}
\end{frame}
%%%%%%%%%%%%%%%%%%
%%%%%%%%%%%%%%%%%%%%
\subsection{Fext and figure}
\begin{frame}
\begin{columns}
    \begin{column}{0.40\textwidth}
        \begin{itemize}
            \item Work hard
            \item Listen carefully
            \item ...
        \end{itemize}
    \end{column}
    \begin{column}{0.6\textwidth}
		\begin{figure}
        \includegraphics[width=0.75\textwidth]{logo.jpg}
				\caption{Be cool}
		\end{figure}
    \end{column}
\end{columns}
\end{frame}
%%%%%%%%%%%%%%%%%%
%%%%%%%%%%
\section{Table}
\begin{frame}
\begin{table}[]
\centering
\caption{My caption}
\label{my-label}
\begin{tabular}{|l|l|l|}
\hline
Name & Sutendet ID & Departmant \\ \hline
     &             &            \\ \hline
     &             &            \\ \hline
\end{tabular}
\end{table}
\end{frame}

% All of the following is optional and typically not needed.
\appendix
\section<presentation>*{\appendixname}
\subsection<presentation>*{For Further Reading}
\begin{frame}

\end{frame}

\section[References]{References}
\begin{frame}[t,allowframebreaks]
 %\frametitle<presentation>{For Further Reading}
%\tiny
%\centerslidesfalse
%\frametitle{REFERENCES}
%\bibliographystyle{amsalpha}

\bibliographystyle{apalike}
%\bibliographystyle{abbrvnat}
\bibliography{mybib}

\end{frame}

\end{document}


